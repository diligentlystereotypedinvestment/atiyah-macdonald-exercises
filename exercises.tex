\documentclass[a4paper]{exam}
\printanswers
\newif\ifhint

\hinttrue

\input{~/templates/math.tex}

\title{Atiyah-MacDonald Solutions}
\author{Vincent Tran}

\begin{document}
\maketitle

\section{Chapter 2}

In Chapter Exercises:

\begin{questions}
	\question 2.2
	\begin{parts}
		\part $\text{Ann}(M + N) = \text{Ann}(M) \cap \text{Ann}(N) $.
		\part $(N:P) = \text{Ann}((N+P) / N) $.
	\end{parts}

	\question 2.15: Let $A,B $ be rings, let $M$ be an $A$-module, $P$ a $B$-module and $N$ an $(A, B)$-bimodule (that is, $N$ is simultaneously an $A$-module and a $B$-module and the two structures are compatible in the sense that $a(xb) = (ax)b$ for all $a \in A, b\in B, x \in N)$. Then $M \otimes _A N$ is naturally a $B$-module, $N \otimes _B P$ an $A$-module, and we have
	\[
		(M \otimes _A N) \otimes _B P \cong M \otimes _A (N \otimes _B P)
	.\]
	\begin{solution}
		% Fix a $p\in P $.
		% Then the map
		% \begin{align*}
		% 	M \times N &\longrightarrow M \otimes _A (N \otimes _B P)\\
		% 	(m,n) &\longmapsto m \otimes_A (n \otimes_B p)
		% \end{align*}
		% is bilinear in A.
		% Hence it induces an $A $-linear map $M \otimes _A N \to M \otimes _A(N \otimes _B P) $ that maps $m \otimes _A n $ to $m \otimes _A (n \otimes _B p) $.

		First we construct the $B $ bilinear map
		\[
			(M \otimes _A N) \times P \to M \otimes _A (N \otimes _B P)
		\]
		that sends $(m \otimes n, p) \to m \otimes (n \otimes p) $.
		The $B $ bilinearity comes from $(b(m \otimes n),p) = (m \otimes nb,p) \mapsto m \otimes (nb \otimes p) = b(m \otimes (n \otimes p)) = m \otimes (b \otimes bp) $, which is also the image of $(m \otimes n, bp) $.
		Hence this induces a unique $B $ linear map
		\[
			(M \otimes_A N) \otimes_B P \to M \otimes _A (N \otimes _B P)
		.\]

		By a symmetric argument, we have a unique $A $ linear map from the other direction, giving us an isomorphism for by tracing where $(m \otimes n) \otimes p $ goes, it goes to $m \otimes (n \otimes p) $ and then to $(m \otimes n) \otimes p $.
	\end{solution}

	\question If $f: A\to B $ is a ring homomorphism and $M $ is a flat $A $-module, then $M_B = B \otimes _A M$ is a flat $B $-module.
	\begin{solution}
		The function $f $ makes $B $ an $A $-algebra.
		Consider $B \otimes_B N \cong N $ by Proposition 2.14.
		Obiously $B $ has an $(A,B) $ bimodule structure since $B $ is an algebra.

		Then given an exact sequence $E $, $E \otimes_A N = E \otimes_A (B \otimes_B N) = (E \otimes_A B) \otimes_B N $ by Exercise 2.15 in the Chapter.
		As $B $ is flat as an $A $-module and $N$ is flat as a $B $-module, we are done.
	\end{solution}
\end{questions}

\begin{questions}
	\question Show that $(\Z / m\Z) \otimes _{\Z} (\Z/n\Z) = 0$ if $m,n $ are coprime.
	\begin{solution}
		Take a bilinear map $f: (\Z / m\Z) \times (\Z / n\Z) $.
		Then by bilinearity, we have $f(mx,y) = f(0,y) + mf(x,y) = f(0,y) $ and $f(x,ny) = f(x,0) + nf(x,y) = f(x,0) $, which imply that $mf(x,y) = 0 $ and $nf(x,y) = 0 $.
		By Bezout's Lemma, we have that there exists $a,b $ s.t. $am+bn = 1 $ as $m,n $ are coprime.
		Thus $f(x,y) = 0 $.
	\end{solution}

	\question Let $A $ be a ring, $M $ an $A $-module. Show that $(A / \mathfrak{a}) \otimes_A M$ is isomorphic to $M / \mathfrak{a}M $. [Tensor the exact sequence $0\to \mathfrak{a}\to A\to A / \mathfrak{a} \to 0 $ with $M $]
	\begin{solution}
		By tensoring with the exact sequence $\mathfrak{a}\to A \to A / \mathfrak{a} \to 0 $, we get
		\[
			0 \to \mathfrak{a}\otimes_A M \to A \otimes_A M \to A / \mathfrak{a} \otimes M \to 0 \tag{Prop 2.8}
		.\]

		Then by Proposition 2.14, $A \otimes M \to M $ is an isomorphism by $a\otimes  $, we have $\mathfrak{a}\otimes M \cong \mathfrak{a}M $ and $A\otimes M \cong M $.
		Hence by commutativity of
		\[
			\begin{tikzcd}
				\mathfrak{a}\otimes M & A \otimes M\\
				\mathfrak{a}M & M
				\arrow[from=1-1,to=1-2]
				\arrow["\cong",from=1-1,to=2-1]
				\arrow[from=2-1,to=2-2]
				\arrow["\cong",from=1-2,to=2-2]
			\end{tikzcd}
		,\]
		(for the commutativity, the definitions of the maps down make it obvious) we have that $\Im(\mathfrak{a}M\to M)= \ker(M\to M / \mathfrak{a}M) = \ker(A / \mathfrak{a}\otimes M)$.

		So we have this diagram
		\[
			\begin{tikzcd}
				&& M / \mathfrak{a}M &\\
				\mathfrak{a}M & M &&0\\
					      && A / \mathfrak{a} \otimes M &
					      \arrow[from=2-1,to=2-2]
					      \arrow[from=2-2,to=1-3]
					      \arrow[from=2-2,to=3-3]
					      \arrow[from=3-3,to=2-4]
					      \arrow[from=1-3,to=2-4]
			\end{tikzcd}
		\]
		By some isomorphism theorem and surjectivity of the last maps, we have that $M / \mathfrak{a}M \cong A / \mathfrak{a} \otimes M $.
	\end{solution}

	\question Let $A $ be a local ring, $M $ and $N $ finitely generated $A $-modules. Prove tha tif $M \otimes_A N = 0 $, then $M = 0 $ or $N = 0 $.
	[Let $\mathfrak{m} $ be the maximal ideal, $k = A / \mathfrak{m} $ the residue field. Let $M_k = k \otimes _A M \cong M / \mathfrak{m}M $ by Exercise 2. By Nakayama's lemma, $M_k = 0 \implies M = 0$. But $M \otimes_A N = 0 \implies (Mo\times _A N)_k = 0\implies M_k \otimes N_k = 0 \implies M_k = 0 $ or $N_k = 0 $, since $M_k,N_k $ are vector spaces over a field.]
	\begin{solution}
		We do as the hint suggests:
		Let $\mathfrak{m} $ be the maximal ideal, $k = A / \mathfrak{m} $ the residue field and define $M_k = k \otimes _A M \cong M / \mathfrak{m}M $ by Exercise 2.

		By Nakayama's lemma, $M_k = 0 \implies M = 0$ since $k \subseteq $ the Jacobson radical, $M_k = 0\implies M = \mathfrak{m}M$, and $M$ is finitely generated.

		Then we have that $M \otimes_A N = 0 \implies (M\otimes _A N)_k = 0$ because

		As the tensor product of vector spaces is just the direct sum, this implies that $0 = k \otimes k \otimes  (M \otimes _A N) = M_k \otimes_A N_k = 0$ by commuting.
		As $A $ bilinear maps on $M_k \times N_k$ are $k $ linear maps on $M_k\times N_k $, we have $M_k \otimes _k N_k = 0$.
		Finally, this implies that $M_k = 0 $ or $N_k = 0 $, since $M_k,N_k $ are vector spaces.
	\end{solution}

	\question Let $M_i (i\in I $) be any family of $A $-modules, and let $M $ be their direct sum. Prove that $M $ is flat $\iff $ each $M_i $ is flat.
	\begin{solution}
		Fix an exact sequence $E $.

		We have that $E \otimes M = E \otimes (\bigoplus M_i) = \bigoplus (E \otimes M_i)$ because each direct sum is finite and hence belongs to a finite direct sum in which we can use Proposition 2.14.
		Then if $E \otimes M $ is exact, so is each coordinate, which gives us the invidiaul $M_i $ is exact.
		If each coordinate is exact then so is $E \otimes M $.
	\end{solution}

	\question Let $A[x] $ be the ring of polynomials in one indeterminate over a ring $A $. Prove that $A[x] $ is a flat $A $-algebra. [Use Exercise 4.]
	\begin{solution}
		Clearly $A[x] = \bigoplus_{i=0}^{\infty} Ax^i$.
		So by the above exercise, it suffices to show that $x^i A $ is flat.
		Say we have a short exact sequence of $A $-modules
		\[
			0 \to B \to C \to D \to 0
		.\]

		Then
		\[
			0 \to B \otimes Ax^i \to C \otimes Ax^i \to D \otimes Ax^i \to 0
		\]
		is exact because tensoring with $Ax^i $ is the same as tensoring with $A $ as bilinear maps $B \times Ax^i $ are bilinear on $B \times A $ and likewise for linear maps.
		So tensoring with $Ax^i $ also induces unique linear maps that make the tensor universal diagram commute, so by uniqueness of the universal property, they are the same.

		Finally tensoring with $A $ is the same as the original module by Proposition 2.14.
		So $Ax^i $ is flat and so is $A[x] $.
	\end{solution}

	\question For any $A$-module, let $M[x]$ denote the set of all polynomials in $x$ with coefficients in $M$, that is to say expressions of the form
	\[
		m_{0}+m_{1}x+\cdots + m_r x^r \tag{$m_i \in M$}
	.\]
	Defining the product of  an element of  $A[x]$ and an element of  $M[x]$ in the obvious  way, show that $M[x]$ is an $A[x]$-module.\\
	Show that $M[x] \cong A[x] \otimes _A M$.
	\begin{solution}
		First, it is an abelian group by commuting and grouping together terms with the same power of $x $.
		Then for the properties of an $A[x] $ module:
		Distributivity holds by simply defining it as so.
		\begin{align*}
			(r+s)m &= (r_{0}+\cdots+r_jx^j+s_{0}+\cdots+s_kx^k)m \\
&= (r_{0}+s_{0}+\cdots+(r_{j+k}+s_{j+k})x^{j+k})m \\
&= (r_{0}+s_{0})m + (r_{1}+s_{1})mx + \cdots \\
&= rm + sm\\
			r(sm) &= r(s_{0}m + s_{1}mx + \cdots + s_kmx^k) \\
&= r(s_{0}m) + \cdots + r(s_kmx^k) \\
&= rs_{0}m + \cdots + rs_kx^km \\
&= (rs_{0}+\cdots+rs_kx^k)m \\
&= (rs)m\\
			1m = m
		.\end{align*}
		Hence $M[x]$ is an $A[x] $ module.

		We use the universal property.
		Say we have a bilinear map
		\[
		\begin{tikzcd}
			A[x] \times M & M[x]\\
		 & B
		\arrow[from=1-1,to=1-2]
		\arrow["f",from=1-1,to=2-2]
		\end{tikzcd}
		\]
		where the top map takes $(a(x),m) \to a(x)m $.
		Then we have the unique linear map $\hat{f}: M[x] \to B$ that takes $m_{0}+m_{1}x+\cdots + m_rx^r $ to $f(1,m_{0})+f(x,m_{1})+\cdots+f(x^r,m_r) $.
		This is linear because of linearity of $f $ in $M $.
		It is unique because we have a basis that uniquely determines the map by linearity, and the bases have to be mapped to the things that generate this map.

		Hence by the universal property, $M[x] \cong A[x] \otimes M $.
	\end{solution}

	\question Let $\mathfrak{p} $ be a prime ideal in $A $. Show that $\mathfrak{p}[x] $ is a prime ideal in $A[x] $. If $\mathfrak{m} $ is a maximal ideal in $A $, is $\mathfrak{m}[x] $ a maximal ieal in $A[x] $?
	\begin{solution}
		It is clear that $A[x] / \mathfrak{p}[x] \cong (A / \mathfrak{p})[x]$.
		As $A / \mathfrak{p} $ is an integral domain by primality of $\mathfrak{p} $, $(A / \mathfrak{p})[x] $ is an integral domain (look at leading coefficients) and thus $p[x]$ is prime.

		Similarly with $\mathfrak{m} $, $A[x] / \mathfrak{m}[x] \cong (A / \mathfrak{m})[x]$.
		Then $A / \mathfrak{m} $ is a field, and clearly $(A / \mathfrak{m})[x] $ is not a field.
	\end{solution}

	\question
	\begin{parts}
		\part If $M $ and $N $ are flat $A $-modules, then so is $M \otimes _A N $.
		\begin{solution}
			Let $E $ be an exact sequence.
			Then $E \otimes_A M $ is exact by $M $ being flat, and hence $(E \otimes_A) \otimes_A N $ is exact.
			By Proposition 2.14, this sequence equals $E \otimes_A (M \otimes_A N) $, so $M \otimes _A N $ is flat.
		\end{solution}
		\part If $B $ is a flat $A $-algebra and $N $ is a flat $B $-module, then $N $ is flat as an $A $-module.
		\begin{solution}
			Consider $B \otimes_B N \cong N $ by Proposition 2.14.
			Then $N $ is flat as an $A $-module by Exercise 2.20 in the Chapter.
		\end{solution}
	\end{parts}

	\question Let $0\to M'\to M\to M''\to 0 $ be an exact sequence of $A$-modules. If $M' $ and $M'' $ are finitely generated, then so is $M $.
	\begin{solution}
		As the last map is surjective, the preimage of $M'' $ is $M $, so the preimage of the generators of $M'' $ and the kernel generate $M $.
		But the kernel is finitely generated as it is the image of $M' $, so $M $ is finitely generated.
	\end{solution}

	\question Let $A $ be a ring, $\mathfrak{a}$ an ideal contained in the Jacobson radical of $A $; let $M $ be an $A $-module and $N $ a finitely generated $A$-module, and let $u: M\to N $ be a homomorphism. If the induced homomorphism $M / \mathfrak{a}M \to N / \mathfrak{a}N $ is surjective, then $u $ is surjective.
	\begin{solution}
		It suffices to show that $N = \mathfrak{a}N + u(M) $ by Corollary 2.7.
		Clearly $\mathfrak{a}N + u(M) \subseteq N$ by definitions.

		Let $\phi_N$ be the quotient map $N\to N / \mathfrak{a}N $ and $\phi_M $ be the map $M \to M / \mathfrak{a}M $.
		Then because $\hat{u}  $ is induced by $\phi \circ u $, $\phi_N \circ u = \hat{u}\phi_M  $.
		As both $\hat{u}  $ and $\phi _M $ are surjective, the LHS is too.
		Hence for every element $n $ of $N $, by the surjectivity of $\phi_N $, there is an element in $u(M) $ s.t. $\phi _N $ of it equals $n $.
		Thus $u(M) + \ker \phi _N = u(M) + \mathfrak{a}N = N$.
	\end{solution}

	\question Let $A $ be a ring $\ne 0 $. Show that $A^m \cong A^n \implies m = n $.
	\ifhint
		Let $\mathfrak{m} $ be a maximal ideal of $A $ and let $\phi : A^m \to A^n $ be an isomorphism.
		Then $1 \otimes \phi: A / \mathfrak{m} \otimes A^m \to A / \mathfrak{m} \otimes A^n$ is an ismorphism of vector space of dimensions $m $ and $n $ over the field $k= A / \mathfrak{m} $.
		Hece $m=n $. [Cf. Chapter 3, Exercise 15]
	\fi
	\begin{solution}
		Let $\mathfrak{m} $ be a maximal ideal of $A $ and let $\phi : A^m \to A^n $ be an isomorphism.
		Then $1 \otimes \phi: A / \mathfrak{m} \otimes A^m \to A / \mathfrak{m} \otimes A^n$ is an ismorphism of vector space (Exercise 2 for modules over a field).
		But then these vector spaces have to have the same dimension over $A / \mathfrak{m} $, which are $m $ and $n $ respectively.
		So $m=n $.
	\end{solution}
	\begin{parts}
		\part If $\phi : A^m \to A^n$ is surjective, then $m\ge n $.
		\begin{solution}
			Let $\mathfrak{m} $ be a maximal ideal of $A $ and let $\phi : A^m \to A^n $ be a surjection.
			Then $1 \otimes \phi: A / \mathfrak{m} \otimes A^m \to A / \mathfrak{m} \otimes A^n$ is a surjection of vector space (Exercise 2) (surjectivity from right exactness of tensoring (Proposition 2.18)).
			Then note that their dimensions are $m,n $ respectively.
			As this is a surjective vector space map, $m\ge n $.
		\end{solution}
		\part If $\phi : A^m\to A^n $ is injective, is it always the case that $m\le n $?
		\begin{solution}
			No.
			Consider $A = \Z[x_{1}, \ldots ] $.
			Then consider the map $A \times A \to A $ that maps $(f(x_{1}, \ldots ),g(x_{1}, \ldots )) \to f(x_{1}, x_{3}, \ldots ) + g(x_{2}, x_{4}, \ldots ) $.
			Obviously they are injective as they are inclusions under relabelling.
			Clearly $2 \not\le 1 $.
		\end{solution}
	\end{parts}

	\question Let $M $ be a finitely generated $A $-module and $\phi :M\to A^n $ a surjective homomorphism. Show that $\ker \phi  $ is finitely generated.
	\ifhint
		Let $e_{1}, \ldots , e_n $ be a basis of $A^n $ and choose $u_i \in M $ such that $\phi (u_i) = e_i (1\le i\le n)$. Show that $M $ is the direct sum of $\ker \phi  $ and the submodule generated by $u_1, \ldots , u_n $.
	\fi
	\begin{solution}
		We can see that $M $ is generated by picking a representative in the preimage of the basis of $A^n $ and the kernel since the quotient is surjective, so every element in $M $ is an element in $A^n $ up to the kernel of the map.
		This then forms a basis because the preimage of the basis is a basis (otherwise push forward a relation), and the kernel and the basis have no relations because otherwise the basis would be in the kernel.

		Hence $M = \ker \phi \oplus A^n $, which implies that $\ker \phi  $ is finitely generated, otherwise $M $ wouldn't be finitely generated.
	\end{solution}

	\question Let $f: A\to B $ be a ring homomorphism, and let $N$ be a $B$-module. Regarding $N$ as an $A$-module by restriction of scalars, form the $B$-module $N_B = B \otimes _A N$. Show that the homomorphism $g: N \to N_B$ which maps $y$ to $1 \otimes  y$ is injective and that $g(N)$ is a direct summand of $N_B$.
	\ifhint
		Define $p: N_B \to N_b$ by $p(b \otimes y) = by$, and show that $N_B = \Im (g) \oplus \ker (p)$.
	\fi
	\begin{solution}
		We can see that $p(g(n)) = n \forall n \in N $, which is injective.
		Hence $g $ is.

		To show that it is the direct sum, we do as the hint suggests and realize that $g\circ p $ is the identity map on elements not in $\ker p $ because $B $ is generated as a $B $-module by $1 $, so we have generators of $N_B $ being of the form $1 \otimes q$.
		It easily follows that $gp(1 \otimes q) = 1 \otimes q $, $g$ is $B $ linear.

		Thus every element of $N_B $ is either in the image of $g $ or in the kernel of $p $.
		Then to show they are independent, suppose we have a non-trivial relation $\sum b_i \otimes y_i + 1 \otimes y \in \ker p + \Im g $ equalling 0.
		Then $p(\sum b_i \otimes y_i + 1 \otimes y) = 0 + p(1 \otimes y) = y \ne 0$ otherwise it would be a trivial relation.
	\end{solution}

	\question A partially ordered set $I$ is said to be a directed set if for each pair $i, j$ in $I$ there exists $k \in I$ such that $i\le k $ and $j\le k $.\\
	Let $A$ be a ring, let $I$ be a directed set and let $(M_i)'_{i\in I}$ be a family of $A$-modules indexed by·$I$. For each pair $i,j$ in $I$ such that $i \le j$, let $\mu _{ij}: M_i \to M_j $ be an $A$-homomorphism, and suppose that the following axioms are satisfied:
	\begin{enumerate}
		\item $\mu_{ii} $ is the identity mapping of $M_i$ for all $i\in I $;
		\item $\mu _{ik} = \mu _{jk}\circ \mu _{ij}$ whenever $i\le j \le k $.
	\end{enumerate}
	Then the modules $M_i$ and homomorphisms $\mu_{ij} $ are said to form a direct system $M = (M_i, \mu _{ij})$ over the directed set $I$.

	We shall construct an $A$-module $M$ called the \texttt{direct limit} of the direct system $M$. Let $C$ be the direct sum of the $M_i$, and identify each module $M_i$ with its canonical image in $C$. Let $D$ be the submodule of $C$ generated by all elements of the form $x_i - \mu _{ij}(x_i) $ where $i\le j $ and $x_i\in M_i $. Let $M = C / D $, let $\mu :C\to M $ be the projection and let $\mu _i $ be the restriction of $M_i $.

	The module $M$, or more correctly the pair consisting of $M$ and the family of homomorphisms $\mu _i:M_i \to M $, is called the direct limit of the direct system $M$, and is written $\lim_{\rightarrow} M_i $ From the construction it is clear that $\mu _i = \mu _j \circ \mu _{ij} $ whenever $i\le j $.
	\begin{solution}
		Let $Q = \{X_i - \mu _{ij}(X_i), X_i \in M_i, i\le j\}$
		We can see that from definition, for $x_i \in M_i $, $\mu _i(x_i) = x_i + Q = x_i - (x_i - \mu_{ij}(x_i)) + Q = \mu_j(\mu_{ij}(x_i))$.
	\end{solution}

	\question In the situation of Exercise 14, show that every element of $M$ can be written in the form $\mu _i(x_i) $ for some $i\in I $ and some $x_i\in M_i $.\\
	Show also that if $\mu _i(x_i) = 0 $ then there exists $j \ge i $ such that $\mu _{ij}(x_i) = 0 $ in $M_j $.
	\begin{solution}
		It suffices to show that $x_j + x_k + Q$ for $x_j\in M_j, x_k\in M_k $ is of the desired form since $M $ are quotient classes of a finite sum.
		We can see that $x_j + x_k + Q = \mu_{j\ell}(x_i) + \mu_{k\ell}(x_k) + Q $ for $j \le \ell $ and $k\le \ell $.
		Then because $\mu_{j\ell}(x_i) + \mu_{k\ell}(x_k)\in M_k $, $\mu_{j\ell}(x_i) + \mu_{k\ell}(x_k) + Q = \mu_k(\mu_{j\ell}(x_i) + \mu_{k\ell}(x_k)) $.

		Since $Q = \mu_i(x_i) = x_i + Q$, there is some finite set of $j_\ell \ge i$ s.t. $x_i = \sum (x_{j_\ell} - \mu_{ij_\ell}(x_{j_\ell})), x_{j\ell} \in M_{i} $.
		Since the $M_i,M_{j\ell} $ are distinct, $\sum \mu _{ij_\ell}(x_{j_\ell}) = 0$.
		% We have that $\mu_i(x_i) = x_i + Q = \mu_{ij}(x_i) + Q $ because, by the definition of directed system, there is a $j $ s.t. $i \le j $.
	\end{solution}

	\question Show that the direct limit is characterized (up to isomorphism) by the following property. Let $N $ be an $A $-module and for each $i\in I $, let $\alpha _i: M_i \to N $ be an $A $-module homomorphism such that $\alpha _i = \alpha _j \circ \mu_{ij} $ whenever $i\le j $. Then there exists a unique homomorphism $\alpha :M\to N $ such that $\alpha _i = \alpha \circ \mu _i $ for all $i\in I $.
	\begin{solution}
		Simply define $\alpha  $ to be $\left(\bigoplus x_i\right) + Q \mapsto \bigoplus \alpha_i(x_i)$.
		This is well-defined because for any $m_i - \mu_{ij}(m_i) \in Q$, this gets mapped to $\alpha _i(m_i) - \alpha_j(\mu _{ij}(m_i)) = \alpha _i(m_i) - \alpha _i(m_i) = 0 $.

		Then this commutes properly because $\alpha (\mu _i(m_i)) = \alpha (m_i+Q) = \alpha _i(m_i)$.

		Finally, this is unique because given another $\alpha' $ with these properties and arbitrary $\bigoplus x_i +Q \in M $, $\alpha' (\bigoplus x_i+Q) = \alpha'(\mu _I(x_I))$ given by Exercise 15.
		Then by definition of $\alpha ' $, $\alpha '(\mu _I(x_I)) = \alpha_I(x_I) = \alpha(\mu _I(x_I)) = \alpha (\bigoplus x_i + Q)$.
		Hence $\alpha '= \alpha  $ for all elements of $M $, and we are done.

		The characterizing up to isomorphism is just a classic universal property argument.
	\end{solution}

	\question Let $(M_i)_{i\in I} $ be a family of submodules of an $A $-module, such that for each pair of indices $i,j $ in $I $, there exists $k\in I $ s.t. $M_i + M_j \subseteq M_k $. Define $i\le j $ to mean $M_i \subseteq M_j $ and let $\mu _{ij}: M_i\to M_j $ be the embedding of $M_i $ in $M_j $. Show that
	\[
		\lim_{\rightarrow} M_i = \sum M_i = \cup M_i
	.\]
	In particular, any $A $-module is the direct limit of its finitely generated submodules.
	\begin{solution}
		Obviously this satisfies the conditions for the direct limit as the maps are just embeddings.
		To show the equality, we can realize $\cup M_i $ as having the properties of the direct limit:
		Say we have a family of maps $\alpha _i $ into an $A $-module $N $ that respect the directed system's maps.

		Then we have a map $\alpha :\cup M_i \to N$ defined by taking an element $m $, finding a $ M_i$ it is in, and mapping it to $\alpha _i(m) $.
		This is well-defined because $\alpha _i $ respects the directed system's maps, those being inclusions.
		Hence it is isomorphic to the direct limit by Exercise 16.

		Since $\{M_i\}   $ is a poset and we have that for every increasing chain, there is a maximal element (namely the union of all the modules in the chain), there is a maximal element $M $ in this set.
		This equals $\cup M_i $, and hence $M \subseteq \sum M_i \subseteq M $.
	\end{solution}

	\question Let $\bm{M} = (M_i, \mu_{ij}), \bm{N}=(N_i,\nu_{ij}) $ be direct systems of $A $-modules over the same directed set. Let $M,N $ be the direct limits and $\mu _i: M_i\to M , \nu_i: N_i \to N$ the associated homomorphisms.\\
	A homomorphism $\bm{\phi}:\bm{M}\to \bm{N} $ is by definition a family of $A $-module homomorphisms $\phi_i:M_i\to N_i $ such that $\phi _j \circ \mu_{ij}=\nu_{ij}\circ \phi_i $ whenever $i\le j $. Show that $\bm{\phi} $ defines a unique homomorphism $\phi = \lim_{\rightarrow} \phi_i: M\to N $ such that $\phi\circ \mu_i = \nu_i\circ \phi_i $ for all $i\in I $.
	\begin{solution}
		We have maps $\psi_i: M_i\to N $ by doing the composition $\nu_i\circ \phi_i $, which commute with the system because $\psi_j(\mu_{ij}) = \nu_j\circ \nu_{ij}\circ \phi_i = \nu_i\circ \phi_i$.
		Hence there is a unique map $M\to N $ that commutes with the system by the characterizing property of the direct limit.
		Since this map commutes with the system, $\phi\circ \mu_i = \psi_i = \nu_i \circ \phi_i $.
	\end{solution}

	\question A sequence of direct systems and homomorphisms
	\[
		\bm{M} \to \bm{N} \to \bm{P}
	\]
	is exact if the corresponding sequence of modules and module homomorphisms is exact for each $i\in I $. Show that the sequence $M\to N\to P $ of direct limits is then exact.
	\begin{solution}
		Let $\mu_{ij},\nu_{ij},\rho_{ij} $ be the maps in the systems, $a_i,b_i $ be the maps $M_i\to N_i $ and $N_i\to P_i $, and let $\mu_{M\cdot}, \nu_{N\cdot}$ be the maps from $M\to \cdot, N\to\cdot $ that are induced by the direct limit property ($\cdot $ will be $N $ or $P $).

		Fix an element $x \in M $.
		By Exercise 15, $x = \mu_i(x_i) $ for some $i $.
		Then we can see that $\mu_{MP} \mu_i = \rho _i b_ia_i $ for all $i $ by commuting properties of the direct limit.
		In particular, $\mu _{MP}(x) = \mu _{MP}\mu_i(x) = \rho_i b_ia_i = 0$ since $a_i,b_i $ are in an exact sequence.

		Finally, we can show that $\ker \nu_{NP} \subseteq \Im \mu_{MN} $ by supposing $\nu_{NP}(x) = 0 $ for $x\in N $.
		Then by Exercise 15, $x = \nu_i(x_i) $ for some $i $.
		By commuting properties, $\nu_{NP}\nu_i = \rho_ib_i $, so $\nu_{NP}(x) = \nu_{NP}\nu_i(x_i) = \rho_ib_i(x_i) = 0 $.
		By exercise 15, if $\rho_i(b_i(x_i)) = 0, $ there exists $j\ge i $ s.t. $\rho_{ij}(b_i(x_i)) = 0 $.
		By commutativity of the diagram, this equals $b_j(\nu_{ij}(x_i)) = 0 $, which by exactness gives us that $\nu_{ij}(x_i) \in \Im a_j $.
		By applying $\nu_j $ to both sides, we can see that $\nu_i(x_i) \in \Im(M_j\to N) $.
		Being in the image of $M_j\to N $ is in the image of $\mu_{MN} $ since, by being the direct limit, $\mu_{MN} $ factors through this map.

		Hence $\ker \nu_{NP} = \Im_{MN} $.

		To understand this proof clearly, let $I $ just be the naturals and draw the commutative diagrams out.
	\end{solution}

	\question Keeping the same notation as in Exercise 14, let $N $ be any $A $-module. Then $(M_i \otimes N, \mu_{ij} \otimes 1) $ is a direct system; let $P = \lim_{\rightarrow}(M_i \otimes N) $ be its direct limit.
	For each $i\in I $, we have a homomorphism $\mu_{i} \otimes 1: M_i \otimes N\to M \otimes N $, hence by Exercise 16 a homomorphism $\psi: P\to M \otimes N $. Show that $\psi $ is an isomorphism, so that
	\[
		\Dlim(M_i \otimes N) \cong (\Dlim M_i) \otimes N
	.\]
	\ifhint
		For each $i\in I $, let $g_i: M_i \times N\to M_i \otimes N $ be the canonical bilinear mapping. Passing to the limit, we obtain a mapping $g: M\times N\to P $. Show that $g $ is $A $-bilinear and hence define a homomorphism $\phi :M \otimes N\to P $. Verify that $\phi \circ \psi $ and $\psi\circ \phi $ are identity mappings.
	\fi
	\begin{solution}
		We show that $M \otimes N $ satisfies the universal property for direct limits.
		Suppose we have maps $\{f_i: M_i \otimes N\to Q\}$.
		Then these lead to bilinear maps $\hat{f_i}: M_i \times N \to Q $
		By direct limit properties, we then have a map $M \times N \to Q $.
		This is bilinear because it commutes with bilinear maps.
		This bilinear map then induces a unique linear map $M \otimes N \to Q $.
		This is the universal property of direct limits, so $M \otimes N \cong \Dlim (M_i \otimes N) $.
	\end{solution}

	\question Let $(A_i)_{i\in I}$ be a family of rings indexed by a directed set $I $, and for each pair $i\le j $ in $I $, let $\alpha _{ij}: A_i\to A_j $ be a ring homomorphism, satisfying conditions (1) and (2) of Exercise 14. Regarding each $A_i $ as a $\Z $-module, we can then form the direct $\text{limit } A = \Dlim A_i $. Show that $A $ inherits a ring structure from the $A_i $ so that the mappings $A_i \to A $ are ring homomorphisms. The ring $A $ is the direct limit of the system $(A_i, \alpha _{ij}) $.\\
	If $A= 0 $ prove that $A_i = 0 $ for some $i\in I $.
	\ifhint
		Remember that all rings have identity elements!
	\fi
	\begin{solution}
		For $a,a' \in \text{limit }A $, define $a\cdot a' $ as $\mu _k(\mu _{ik}(a_i)\mu _{jk}(a_j)) $ where $a = \mu _i(a_i), a' = \mu _j(a_j) $ and $k \ge i,j $, which is well-defined because for other $k' \ge i,j $, we can find a $k'' $ s.t. $\mu _{k'}(\mu _{ik'}(a_i)\mu_{jk'}(a_j)) = \mu_{k''}(\mu_{k'k''}(\mu_{ik'}(a_i)\mu_{jk'}(a_j))) = \mu_{k''}(\mu_{k'k''}(\mu_{ik'}(a_i))\mu_{k'k''}(\mu_{jk'}(a_j))) =\mu _{k''}(\mu_{ik''}(a_i)\mu_{jk''}(a_j)) =\mu _{k''}(\mu_{kk''}(\mu_{ik}(a_i)\mu_{jk}(a_j))) = \mu_k(\mu_{ik}(a_i)\mu_{jk}(a_j)) $.
		This is obviously commutative and has identity over multiplication because it has the domain of a commutative ring and $\mu_{\cdot,\cdot} $ are homomorphisms.

		Next is associativity:
		Let $a = \mu_i(a_i),b=\mu_j(b_j),c=\mu_k(c_k) $ and $\ell\ge i,j,k $.
		\begin{align*}
			(a\cdot b) \cdot c &= (\mu _\ell(\mu _{i\ell}(a_i)\mu_{j\ell}(b_j)))\cdot \mu_{k}(c_k) \\
			\iff\\
			\mu_{\ell}((\mu _{i\ell}(a_i)\mu _{j\ell}(b_j))\mu_{k\ell}(c_k)) &= a\cdot (b\cdot c)
		.\end{align*}

		Finally, distributivity:
		Let $a = \mu_i(a_i),b=\mu_j(b_j),c=\mu_k(c_k) $ and $\ell\ge i,j,k $.
		\begin{align*}
			a (b+c) = \mu _i(a_i)(\mu _j(b_j) + \mu _k(c_k)) = \mu _i(a_i)(\mu_{\ell}\mu_{j\ell}(b_j) + \mu_{\ell}\mu _{k\ell}(c_k)) = \mu _{\ell}(\mu _{i\ell}(a_i)(\mu _{j\ell}(b_j) + \mu _{k\ell}(c_k))) = \mu _{\ell}(\mu _{i\ell}(a_i)\mu _{j\ell}(b_j) + \mu _{i\ell}(a_i)\mu _{k\ell}(c_k)) = \mu _i(a_i)\mu _j(b_j) + \mu _i(a_i)\mu _k(c_k) = ab + ac
		.\end{align*}

		If $A = 0 $, then $\mu _i(1) = 0\implies \mu _{ij}(1) =0 $ by Exercise 15.
		But a ring homomorphism that sends 1 to 0 implies that the ring is 0, so $A_j = 0 $.
	\end{solution}

	\question Let $(A_i, \alpha _{ij}) $ be a direct system of rings and let $\mathfrak{R}_i $ be the nilradical of $A_i $. Show that $\lim_{\rightarrow} \mathfrak{R}_i $ is the nilradical of $\lim_{\rightarrow}A_i $.\\
	If each $A_i $ is an integral domain, then $\lim_{\rightarrow}A_i $ is an integral domain.
	\begin{solution}
		We have the obvious inclusions $\mathfrak{R}_i \to \mathfrak{R}(\Dlim A_i) $ since $A_i\to \text{limit }A $ is a ring homomorphism ($a^n = 0 $ in $A_i $ gets mapped to $a^n = 0 $ in $\text{limit }A $).

		Next we can map $\mathfrak{R}(\Dlim A_i) $ to $\Dlim \mathfrak{R}_i $ as so:
		For any $a^n = 0 \in \text{limit }A $, $a =\mu_i(a_i) $ by Exercise 15, which then gives us $\mu_i(a_i^n) = 0 $.
		By Exercise 15, we then have $\mu _{ij}(a_i^n) = 0 $ in $A_j $.
		Then $\mu _{ij}(a_i)^n = 0 $, giving us an element $\mu_{ij}(a_i) $, which we then map into $\Dlim \mathfrak{R}_i $.

		This is well-defined because we can always commute any choices to the same, largest index ring.
		Next this is a homomorphism because given $a = \mu _k(\mu _{ik}(a_i)), b = \mu _k(\mu _{jk}(b_j))$, $a+b = \mu_k(\mu_{ik}(a_i)+\mu_{jk}(a_j)) \rightarrow \mu_{k\ell}(\mu_{ik}(a_i)+\mu_{jk}(a_j)) \rightarrow \mu_k(\mu_{ik}(a_i) + \mu _{jk}(b_j)) = \mu_i(a_i) + \mu_j(b_j)$, which is what $a,b $ would be mapped to.
		This is just the identity.

		Since there is a homomorphism and an inverse, it is an isomorphism.

		If each $A_i $ is an integral domain, then suppose FTSOC that there is $ab = 0 $ in $\Dlim A_i $, $a,b\ne 0 $.
		Then by Exercise 15, we have $a = \mu_i(a_i),b=\mu_j(b_j)$.
		Hence $\mu _i(a_i)\mu _j(b_j) = 0 = \mu _k(\mu _{ik}(a_i)\mu _{jk}(b_j)) $ for $k\ge i,j $.
		Then by Exercise 15, there is $\ell\ge k $ s.t. $\mu _{k\ell}(\mu _{ik}(a_i)\mu _{jk}(b_j)) = 0 = \mu _{i\ell}(a_i)\mu _{j\ell}(b_j) $.
		But then $A_j $ wouldn't be an integral domain (note that $\mu _{\cdot \ell}(\cdot) \ne 0 $ because if otherwise, then $\mu _\ell(\mu _{\cdot \ell}(\cdot)) = \mu_{\cdot}(\cdot) = 0$, contradicting $a,b $ being non-zero).
	\end{solution}

	\question Let $(B_{\lambda })_{\lambda \in \Lambda} $ be a family of $A $-algebras. For each finite subset of $\Lambda $, let $B_J $ denote the tensor product (over $A $) of the $B_{\lambda} $ for each $\lambda \in J $. If $J' $ is another finite subset of $\Lambda $ and $J\subseteq J' $, there is a canonical $A $-algebra homomorphism $B_J\to B_{J'} $. Let $B $ denote the direct limit of the rings $B_J $ as $J $ runs through all finite subsets of $\Lambda $. The ring $B $ has a natural $A $-algebra structure for which the homomoprhisms $B_J\to B $ are $A$-algebra homomorphisms. The $A $-algebra $B $ is the tensor product of the family $(B_{\lambda})_{\lambda\in \Lambda} $.
	\begin{solution}
		The canonical $A $-algebra homomorphism sends $b\in B_J $ to $b \otimes 1 \otimes 1 \otimes \cdots $ ($|J'| - |J| $ times).
		As $A $-algebras are also rings, the ring $B $ exists by Exercise 21.
		Ring homomorphisms that preserve $A $-module structure are $A $-algebra homomorphisms.
	\end{solution}

	\question In these Exercises it will be assumed that the reader is familiar with the definitio nand basic properties of the Tor functor.

	If $M $ is an $A $-module, the following are equivalent:
	\begin{enumerate}
		\item $M $ is flat;
		\item $\Tor^A_n(M,N) = 0 $ for all $n > 0 $ and all $A $-modules $N $;
		\item $\Tor_1^A(M,N) = 0 $ for all $A $-modules $N $.
	\end{enumerate}
	\ifhint
		To show that (i) $\implies $ (ii), take a free resolution of $N $ and tensor it with $M $. Since $M $ is flat, the resulting sequence is exact and therefore its homology groups, which are the $\Tor^A_n(M,N) $ are zero for $n > 0 $. To show that (iii) $\implies $ (i), let $0\to N' \to N \to N''\to 0 $ be an exact sequence. Then from the $\Tor $ exact sequence,
		\[
			\Tor_1(M,N'') \to M \otimes N' \to M \otimes N \to M \otimes N'' \to 0
		\]
		is exact. Since $\Tor_1(M,N'') = 0 $, it follows that $M $ is flat.
	\fi
	\begin{solution}
		(i) $\implies $ (ii):
		We do as the hint suggests: take a free resolution of $N $.
		Tensor this with $M $.
		As $M $ is flat, this sequence is then exact, so the homology groups are 0.

		Obviously (ii) $\implies $ (iii).

		(iii) $\implies $ (i):
		Take an exact sequence $0\to N' \to N \to N'' \to 0 $.
		Then $\Tor_1^A(M,N'') \to M \otimes N' \to M \otimes N \to M \otimes N'' \to 0 $ is exact.
		As $\Tor_1^A(M,N'') = 0 $, $M $ is flat.
	\end{solution}

	\question Let $0\to N'\to N\to N''\to 0 $ be an exact sequence with $N'' $ flat. Then $N' $ is flat $\iff N $ is flat.
	\ifhint
		Use Exercise 24 and the Tor exact sequence.
	\fi
	\begin{solution}
		By the Tor exact sequence, we have
		\[
			\Tor_2^A(M,N'') \to \Tor_1^A(M,N') \to \Tor_1^A(M,N) \to \Tor_1^A(M,N'') \to M \otimes  N' \to M \otimes N \to M \otimes N'' \to 0
		.\]
		As $\Tor_2^A(M,N'') = \Tor_1^A(M,N'') = 0 $ by flatness of $N'' $ and Exercise 24, $\Tor_1^A(M,N') = \Tor_1^A(M, N) $.
		By Exercise 24, this means that $N $ is flat iff $N' $ is flat.
	\end{solution}

	\question Let $N $ be an $A $-module. Then $N $ is flat $\iff \Tor_1(A / \mathfrak{a}, N) = 0 $ for all finitely generated ideals $\mathfrak{a}$ in $A $
	\ifhint
		Show first that $N$ is flat if $\Tor_1(M, N) = 0$ for all finitely generated $A$-modules $M$, by using (2.19). If $M$ is finitely generated, let $x_1, \cdots , x_n$ be a set of generators of $M$, and let $M$, be the submodule generated by $x_{1}, \ldots ,x_i$. By considering the successive quotients $M_i/M_{i-1}$ and using Exercise 25, deduce that $N$ is fiat if $\Tor_l(M, N) = 0$ for all cyclic $A$-modules $M$, i.e., all $M$ generated by a single element, and therefore of the form $A/\mathfrak{a}$ for some ideal $\mathfrak{a} $. Finally use (2.19) again to reduce to the case where $\mathfrak{a} $ is a finitely generated ideal.
		\fi
	\begin{solution}
		$\implies)$ is obvious by Exercise 24.

		$\Leftarrow) $:
		\begin{lem}
			Assuming the RHS, then $\Tor(A / \mathfrak{b},N) = 0 $ for \textbf{all} $\mathfrak{b} $.
		\end{lem}
		\begin{proof}
			Take the system of finitely generated submodules of $\mathfrak{b} $ call it $F $, the system associated to that index of just $A $ call it $\bm{A} $, and the system $A / \mathfrak{b}_i $ for $b_i \in F$ call it $Q $.
			As each of these has an exact sequence $0\to\mathfrak{b} \to A \to A / \mathfrak{b}\to 0$, we have an exact sequence of systems by Exercise 19, giving us an exact sequence $0\to \Dlim F \to \Dlim \bm{A} \to \Dlim Q \to 0$.
			As tensors commute with direct limits (Exercise 20), we have $0\to \Dlim F \otimes N \to \Dlim \bm{A} \otimes N \to \Dlim Q \otimes N \to 0 $.

			By Exercise 17, $\Dlim F = \mathfrak{b} $.
			We also have that $\Dlim Q = A / \mathfrak{b} $ because all the maps in the system $Q $ have kernels contained in $B $, so by the universal property of the quotient it induces a unique map from $A / \mathfrak{b} $that commutes with the system, so by the universal property of the direct limit, it is the direct limit.
			So we have the exact sequence
			\[
				0 \to \mathfrak{b} \otimes N \to A \otimes N \to A / \mathfrak{b} \otimes N\to 0
			.\]
		\end{proof}

		First we can note that $N $ is flat if $\Tor_1(M,N) = 0 $ for all finitely generated $A $-modules $M $ by Proposition 2.19.
		Then fix a finitely generated $M $ generated by $x_i $ and define $M_i = \{x_{1},\ldots, x_i\}   $.
		Also define the map $f_i: A \to M_i / M_{i-1} $ by sending $a \in A $ to $ax_i + M_{i-1} $.
		This is surjective as $M_i $ is generated by $x_{1}, \ldots , x_i $.
		As such, $\ker f_i $ is an ideal of $A $.
		Hence $M_i / M_{i-1} \cong A / \ker f_i$
		So by considering the exact sequence $0 \to M_{i-1} \to M_{i}\to M_i / M_{i-1} \cong A / \ker f_i \to 0$, we can see that we get the Tor sequence
		\[
			\Tor_1(M_{i-1},N) \to \Tor_1(M_i,N) \to \Tor_1(A / \ker f_i, N)
		.\]

		Assume for induction that $\Tor_1(M_{i-1},N) = 0 $.
		Then by the lemma above, $\Tor_1(A / \ker f_i,N) = 0 $.
		Thus $\Tor_{1}(M_i,N) = 0 $.
		Obviously $\Tor_{1}(M_{0},N) = 0 $.
		Thus $\Tor_{1}(M,N) = 0 $ for all finitely generated $M $, allowing us to use Proposition 2.19 to finish.
	\end{solution}

	\question A ring $A $ is absolutely flat if every $A $-module is flat. Prove that the following are equivalent:
	\begin{enumerate}
		\item $A $ is absolutely flat.
		\item Every principal ideal is idempotent.
		\item Every finitely generated ideal is a direct summand of $A $.
	\end{enumerate}
	\ifhint
		Let $x\in A $. Then $A / (x) $ is a flat $A $-module, hence in the diagram
		\[
		\begin{tikzcd}
			(x) \otimes A & (x) \otimes A / (x)\ \
				A & A / (x)
		\arrow[from=1-1,to=1-2]
		\arrow["\beta",from=1-1,to=2-1]
		\arrow["\alpha",from=2-1,to=2-2]
		\arrow[from=1-2,to=2-2]
		\end{tikzcd}
		\]
		the mapping $\alpha  $ is injective. Hence $\Im(\beta ) = 0 $, hence $(x) = (x)^2 $. (ii) $\implies $ (iii): Let $x\in A $. Then $x = ax^2 $ for some $a\in A $, hence $e = ax $ is idempotent and we have $(e) = (x) $. Now if $e,f $ are idempotents, then $(e,f) = (e+f-ef) $. Hence every finitely generated ideal is principal, and generated by an idempotent $e $, hence is a direct summand because $A = (e) \oplus (e-e) $. (iii) $\implies $ (i): Use the criterion of Exercise 26.
	\fi
	\begin{solution}
		(i) $\implies $ (ii): Since $A / (x) $ is an $A $-module, it is flat.
		Thus the injectivity of $(x) \to A $ makes the map $(x) \otimes A / (x) \to A \otimes A / (x) \cong A / (x) $ injective.
		This map takes $x \otimes [a] \mapsto x \otimes [a] \mapsto [xa] = 0$ (middle map is due to Proposition 2.19).
		As it is an injective zero map, $(x) \otimes A / (x) = 0$, and by Exercise 2, $(x) \otimes A / (x) \cong (x) / (x)^2 $.
		Thus $(x) = (x)^2 $.

		(ii) $\implies $ (iii):
		As the hint does:
		Let $x\in A $.
		Then $x = ax^2 $ for some $a\in A $, hence $e = ax $ is idempotent and we have $(e) = (x) $.
		For idempotents $e,f $, $(e,f) = (e+f -ef) $ because $e(e+f-ef) = e + ef - ef = e $ and $f(e+f-ef) = ef + f - ef = f $.
		Thus every finitely generated ideal is principal by finding idempotents for every generator in the ideal and then reducing them pairwise as so.
		As such, $A = (e) \oplus (1-e) $ (note that $(1-e)^2 = (1-e) $, so they are independent).

		(iii) $\implies $ (i):
		It suffices to satisfy the conditions in Exercise 26 for all $N $.
		Take an exact sequence $0\to N' \to N \to N'' \to 0$.
		Then we have the sequence
		\[
			\Tor_1(A / \mathfrak{a},N'') \to N' \otimes A / \mathfrak{a} \to N \otimes A / \mathfrak{a} \to N'' \otimes A / \mathfrak{a}\to 0
		.\]
		By Exercise 2, $N' \otimes A / \mathfrak{a} \cong N' / \mathfrak{a}N' \cong \mathfrak{b}N' $ as we assume that $A $ is a direct sum of f.g. ideals (namely let $A = \mathfrak{a}\oplus \mathfrak{b} $).
		Then the map $N' \otimes A / \mathfrak{a} \to N \otimes A / \mathfrak{a} $ is the map $\mathfrak{b}N' \to \mathfrak{b}N $, which is injective as they are simply restrictions of the injective map $N' \to N $.
		Thus $\Tor_1(A / \mathfrak{a}, N'') = 0 $.
		As we can always realize $N $ as the tail of an exact sequence (simply take $0\to 0 \to N \to N \to 0$, we are done.
	\end{solution}

	\question A Boolean ring is absolutely flat. The ring of Chapter 1, Exercise 7 is absolutely flat. Every homomorphic image of an absolutely fiat ring is absolutely flat. If a local ring is absolutely flat, then it is a field. If $A$ is absolutely flat, every non-unit in $A$ is a zero-divisor.
	\begin{solution}
		By definition, all principal ideals are idempotent in a Boolean ring, so by Exercise 27 we are done.

		The ring in Chapter 1 Exercise 7 is absolutely flat because all principal ideals are idempotent: $(x)^2= (x^2) = (x) $ because $x^{2n} = x \in (x^2) $.

		Say we have $f $ a homomorphism from an absolutely flat ring $R $.
		Then every principal ideal in the image is generated by $f(a) $, and $(a^2) = (a)$ by Exercise 27.
		Hence $(f(a))^2 = (f(a)^2) = (f(a^2)) = (f(a)) $.

		Fix an absolutely flat local ring $R $.
		By Exercise 27, every principal ideal of $R $ is idempotent, so $(x^2) = (x) \forall x \in R $.
		Hence $x = rx^2, r \in R $.
		Thus $rx = r^2x^2 = (rx)^2 \implies rx $ is idempotent.
		But by Exercise 12 of Chapter 1, $rx = 0 $ or $1 $.
		Thus $(x) = 0 $ or $1 $, which implies that it is a field.

		If $A $ is absolutely flat, then take a non-unit $x $.
		We have that $(x)^2 = (x) $, so $x \in (x^2) \implies rx^2 = x $ for some $r $.
		Thus $x(rx-1) = 0 \implies $ $x $ is a zero-divisor.
	\end{solution}
\end{questions}

\end{document}
