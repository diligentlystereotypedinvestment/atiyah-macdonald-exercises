% root = exercises.tex
\newif\ifhint
\hinttrue
\section{Primary Decomposition}

\begin{exercise}
	If an ideal $\mathfrak{a}$ has a primary decomposition, then $\Spec(A/\mathfrak{a})$ has only finitely many irreducible components.
\end{exercise}
\begin{proof}
	By the First Uniqueness Theorem, the associated primes to $\mathfrak{a} $ is independent of the decomposition.
	As $\mathfrak{a} $ being decomposable implies that $\mathfrak{a} $ is a finite intersection, the number of associated primes must be finite.
	Finally, because irreducible components of $\Spec(A / \mathfrak{a}) $ correspond to minimal primes belonging to $\mathfrak{a} $ by the remark after Prop 4.6, this implies that there are a finite number of irreducible components.
\end{proof}

\begin{exercise}
	If $\mathfrak{a} = r(\mathfrak{a})$, then $\mathfrak{a}$ has no embedded prime ideals. 
\end{exercise}
\begin{proof}
	I'm going to assume that $\mathfrak{a} $ is decomposable, because otherwise the definition in the chapter doesn't make sense.
	Thus decompose $\mathfrak{a} $ as a minimal primary decomposition $\cap^n \mathfrak{q}_i $ and let $r(\mathfrak{q}_i) = \mathfrak{p}_i $.
	Then $\mathfrak{a} = r(\mathfrak{a}) = r(\cap^n \mathfrak{q}_i) = \cap^n r(\mathfrak{q}_i) = \cap^n \mathfrak{p}_i$.
	If there were any embedded prime ideals among the $\mathfrak{p}_i $, then we could eliminate the term and see that $\cap^n r(\mathfrak{q}_i) $ is not minimal, a contradiction.
	Thus there are no embedded primes.
\end{proof}

\begin{exercise}
	If $A$ is absolutely flat, every primary ideal is maximal. 
\end{exercise}
\begin{proof}
	Let $\mathfrak{q} $ be a primary ideal.
	Our strategy will be to show that $A / \mathfrak{q} $ is a field.
	Then $A / \mathfrak{q} $ is absolutely flat by Exercise 2.28 (homomorphic image).

	By Exercise 2.28, all non-units are zero divisors, and because $\mathfrak{q} $ is primary, all zero-divisors are nilpotents.
	By Exercise 1.10, a ring where every element is either nilpotent or a unit is local.
	By Exercise 2.28, this implies that $A / \mathfrak{q} $ is a field.
\end{proof}

\begin{exercise}
	In the polynomial ring $\Z[t] $, the ideal $\mathfrak{m}= (2,t) $ is maximal and the ideal $\mathfrak{q} = (4,t) $ is $\mathfrak{m} $-primary, but is not a power of $\mathfrak{m} $.
\end{exercise}
\begin{proof}
	First we can see that $\mathfrak{m} $ is maximal because $\Z[t] / \mathfrak{m} \cong \Z_2 $, which is a field.
	Then, $\mathfrak{q} $ is $\mathfrak{m} $-primary because 

	(1) it is a primary ideal for if $ab = 0 \in \Z[t] / \mathfrak{q} $, then because $\Z[t] / \mathfrak{q} \cong \Z_4 $, we either have one of the terms equivalent to $4 $ in $\Z_4 $ or both terms are equivalent to $2 $.
	The former case is $ab \in \mathfrak{q} \implies a \in \mathfrak{q} $ and the latter is when $a\notin \mathfrak{q} \implies b^n \in \mathfrak{q} $, with $n=2 $ here.

	(2) to check that $\sqrt{\mathfrak{q}}  = \mathfrak{m}$, it suffices to show that $(\Z[t] / \mathfrak{q}) / \mathfrak{N} \cong \Z[t] / \mathfrak{m} $ with $\mathfrak{N} $ the nilradical of $\Z[t] / \mathfrak{q} $.
	We can see that the nilradical of $\Z[t] / \mathfrak{q} \cong \Z_4 $ is isomorphic to the ideal $(2) $ in $\Z_4 $, and $(\Z[t] / \mathfrak{q}) / \mathfrak{N} \cong \Z_4 / (2) \cong \Z_2 \cong \Z[t] / \mathfrak{m}$.

	Finally, suppose FTSOC that $(4,t) = (2,t)^n $.
	If $n>2 $, $4 \notin (2,t) $.
	Thus $n=2 $ as $n=1 $ is obviously eliminated.
	But $(2,t)^2 = (4,2t,t^2) $, which doesn't contain $t $.
\end{proof}

\begin{exercise}
	In the polynomial ring $K[x,y,z]$ where $K$ is a field and $x, y,z$ are independent indeterminates, let $\mathfrak{p}_1 = (x,y)$, $\mathfrak{p}_2 = (x,z)$, $\mathfrak{m} = (x,y,z)$; $\mathfrak{p}_1$ and $\mathfrak{p}_2$ are prime, and $\mathfrak{m}$ is maximal. Let $\mathfrak{a} = \mathfrak{p}_1\mathfrak{p}_2$. Show that $\mathfrak{a} = \mathfrak{p}_1 \cap \mathfrak{p}_2 \cap \mathfrak{m}^2$ is a reduced primary decomposition of $\mathfrak{a}$. Which components are isolated and which are embedded? 
\end{exercise}
\begin{proof}
	Obviously $\mathfrak{a} \subseteq \mathfrak{p}_1 \cap \mathfrak{p}_2 \cap \mathfrak{m}^2 $ as $\mathfrak{a} = (x^2,xz,yx,yz) $.
	Now suppose we have $a \in \mathfrak{p}_1 \cap \mathfrak{p}_2 \cap \mathfrak{m}^2 $.
	Then because $a \in \mathfrak{p}_1 $, $a = xp(x,y,z)+yq(x,y,z) $.
	Because $a \in \mathfrak{p}_2 $ and $xp(x,y,z) \in \mathfrak{p}_2 $ and $y$ is prime in $K[x,y,z] $, we must have that either $x| q(x,y,z) $ or $z | q(x,y,z) $.
	In either case, $x^2 $ or $yz | yq(x,y,z) $, so that term is in $\mathfrak{p}_1 \mathfrak{p}_2$.

	Thus all we need to show now is that $xp(x,y,z) \in \mathfrak{p}_1 \mathfrak{p}_2 $.
	From the above, we can also conclude that $yq(x,y,z) \in \mathfrak{m}^2 $.
	Thus we know that $xp(x,y,z) \in \mathfrak{m}^2 $, hence either $x,y,z $ divides $p $.
	In all three cases, this puts $xp $ in $\mathfrak{p}_1 \mathfrak{p}_2 $.
	Hence $a \in \mathfrak{a} $.

	Because $\sqrt{m^2}  = \mathfrak{m} $ as $\mathfrak{m} $ is prime, $\mathfrak{m} $ is an associated prime.
	As $\mathfrak{p}_1, \mathfrak{p}_2 $ are prime, they are associated primes.
	Clearly $\mathfrak{p}_1,\mathfrak{p}_2 $ are minimal and contain $\mathfrak{m} $, so $\mathfrak{p}_i $ are isolated and $\mathfrak{m} $ is embedded.
\end{proof}

\begin{exercise}
	Let $X\!$ be an infinite compact Hausdorff space, $C(X)$ the ring of real-valued continuous functions on $X\!$ (Chapter 1, Exercise 26). Is the zero ideal decomposable in this ring? 
\end{exercise}
\begin{proof}
	No.
	Recall that by Exercise 1.16 that every maximal ideal is of the form $\mathfrak{m}_x = \{f \in C(X) | f(x) = 0\}   $.
	First we show that every primary ideal is contained in exactly one maximal ideal.
	Suppose we have primary $\mathfrak{p} \subseteq \mathfrak{m}_x \cap \mathfrak{m}_y$.

	Because $X $ is Hausdorff, there is open disjoint neighborhoods $U,V $ such that $x \in U $ and $y \in V $.
	By Urysohn's Lemma, we have $f_x, f_y \in C(X)$ such that $f_x(x) = 1, f_x(U^c) = 0, f_y(y) = 1, f_y(V^c) = 0$.
	Then $f_xf_y = 0 $ because $f_x $ is non-zero on $U $ and $f_y $ is non-zero on $V $, so their product is non-zero on $U\cap V = \emptyset$.
	Because $f_x(x) = 1 $ and $f_y(y) = 1 $, neither are in $\mathfrak{p} $.
	But this contradicts $\mathfrak{p} $ being primary.

	Now suppose FTSOC that $(0) = \cap^n \mathfrak{q}_i $.
	Let $\mathfrak{q}_i \subseteq \mathfrak{m}_{x_i} $.
	Now take a point $x \notin \{x_i\}   $.
	By Urysohn's Lemma, there is $\delta_i$ that vanishes on $x_i $ and not $x $.
	Let $f_i $ be the product of an element in $\mathfrak{q}_i $ that doesn't vanish on $x $ (because $\mathfrak{q}_i \not \subseteq \mathfrak{m}_x $) with $\delta_i $.
	Then $\prod f_i \in \cap \mathfrak{q}_i $ but isn't $0 $ as it doesn't vanish on $x $.
	This contradicts $(0) = \cap^n \mathfrak{q}_i $.
\end{proof}

\begin{exercise}
	Let $A$ be a ring an let $A[x] $ denote the ring of polynomials in one indeterminate over $A $. For each ideal $\mathfrak{a} $ of $A $, let $\mathfrak{a}[x] $ denote the set of all polynomials in $A[x] $ with coefficients in $\mathfrak{a} $.
	\begin{enumerate}
		\item $\mathfrak{a}[x]$ is the extension of $\mathfrak{a}$ to $A[x]$.
		\item If $\mathfrak{p}$ is a prime ideal in $A$, then $\mathfrak{p}[x]$ is a prime ideal in $A[x]$.
		\item If $\mathfrak{q}$ is a $\mathfrak{p}$-primary ideal in $A$, then $\mathfrak{q}[x]$ is a $\mathfrak{p}[x]$-primary ideal in $A[x]$.
		\item If $\mathfrak{a} = \cap_{i=1}^n \mathfrak{q}_i$ is a minimal primary decomposition in $A$, then $\mathfrak{a}[x] = \cap_{i=1}^n \mathfrak{q}_i[x]$ is a minimal primary decomposition in $A[x]$.
		\item If $\mathfrak{p}$ is a minimal prime ideal of $\mathfrak{a}$, then $\mathfrak{p}[x]$ is a minimal prime ideal of $\mathfrak{a}[x]$.
	\end{enumerate}
\end{exercise}
\begin{proof}
	i) The image of $\mathfrak{a} $ in $A[x] $ is $\mathfrak{a} $.
	The ideal generated by it includes $ax^n $ for $a\in \mathfrak{a} $ and natural $n $.
	Thus all polynomials with coefficients in $\mathfrak{a} $ are in the ideal.
	Finally, all elements of the ideal generated by $\mathfrak{a} $ in $A[x] $ have coefficients in $\mathfrak{a} $ by definition (finite sums of products of elements of $A[x] $ with elements of $\mathfrak{a} $).

	ii) We have that $A[x] / \mathfrak{p}[x] \cong (A / \mathfrak{p})[x] $ because each element of $A[x] / \mathfrak{p}[x] $ is a polynomial with coefficients in $A / \mathfrak{p} $.
	A polynomial ring over an integral domain $(A / \mathfrak{p})[x] $ is an integral domain.

	iii) Consider $A[x] / \mathfrak{q}[x] $.
	By the above, this is isomorphic to $(A / \mathfrak{q})[x] $.
	Now suppose we have a zero divisor in this ring, say $fg=0 $.
	Then by chapter 1 exercise 2, we have $a \in A / \mathfrak{q} $ such that $af = 0 $.
	Thus every coefficient of $f $ is a zero divisor in $A / \mathfrak{q} $.
	Therefore every coefficient is nilpotent, as $\mathfrak{q} $ is primary.
	By Chapter 1 Exercise 2 again, this makes $f $ nilpotent.

	Next we can see that $\mathfrak{p}[x] \subseteq r(\mathfrak{q}[x])$ because $\mathfrak{q}\subseteq \mathfrak{q}[x] $ and $r(\mathfrak{q}[x]) $ is an ideal in $A[x] $.
	Finally, we can note that $\mathfrak{q}[x] \subseteq \mathfrak{p}[x] $ because $\mathfrak{q} \subseteq \mathfrak{p} $, and as $r(\mathfrak{q}[x]) $ is the smallest prime ideal containing $\mathfrak{q}[x] $, it must equal $\mathfrak{p}[x] $.

	\begin{lem}
		$\mathfrak{p} \subseteq \mathfrak{q} \iff \mathfrak{p}[x] \subseteq \mathfrak{q}[x] $
	\end{lem}
	\begin{proof}
		If $\mathfrak{p} \subseteq \mathfrak{q} $, any polynomial with coefficients in $\mathfrak{p} $ have coefficients in $\mathfrak{q} $.

		If $\mathfrak{p}[x] \subseteq \mathfrak{q}[x] $, then $\mathfrak{p} \subseteq \mathfrak{p}[x]$ gives us that $\mathfrak{p}\subseteq \mathfrak{q} $ ($\mathfrak{q} $ is the degree zero component).
	\end{proof}

	iv) We can see that $\cap \mathfrak{q}_i[x] = (\cap \mathfrak{q}_i)[x] $ because a polynomial in all $\mathfrak{q}_i[x] $ has coefficients in all $\mathfrak{q}_i $.
	By iii), all we need to show is that $\cap \mathfrak{q}_i[x] $ is a minimal primary decomposition.
	Condition i) is clearly met because the degree $0$ components are distinct.
	The conditions for ii) are satisfied because of the above lemma and the fact that $(\mathfrak{q}_i)[x] = \cap \mathfrak{q}_i[x]$.

	v) This is because of iv) and the lemma above.
\end{proof}

\begin{exercise}
	Let $k$ be a field. Show that in the polynomial ring $k[x_1,\ldots,x_n]$ the ideals $\mathfrak{p}_i = (x_1,\ldots,x_i)$ $(1 \leq i \leq n)$ are prime and all their powers are primary. \ifhint Use Exercise 7.
\fi
\end{exercise}
\begin{proof}
	Clearly $k[x_{1}, \ldots ,x_n] / \mathfrak{p}_i $ is prime because this is isomorphic to $k[x_{i+1}, \ldots, x_n] $, an integral domain.
	Let $\mathfrak{q}_i^m = \mathfrak{p}_i^m \cap k[x_{1}, \ldots ,x_i] $.
	Then $\mathfrak{p}_i^m = (\mathfrak{q}_i^m)^e = \mathfrak{q}_i^m[x_{i+1}, \ldots , x_n]$.
	By 7iii), it then suffices to show that $\mathfrak{q}_i^m$ is $\mathfrak{q}_i$ primary (as this will then make $\mathfrak{q}_i^m[x_{i+1}, \ldots ,x_n]$ primary).
	By discussion in the chapter, $r(\mathfrak{q}_i^m) = \mathfrak{q}_i $.
	So all we need to do is to show that $\mathfrak{q}_i^m $ is primary in $k[x_{1}, \ldots ,x_i] $.

	Suppose we have $ab = 0 $ in $k[x_{1}, \ldots ,x_i] / \mathfrak{q}_i^m$.
	We can see that $a $ or $b $ has no degree 0 component, because otherwise we would have a degree $0 $ element that can't be cancelled out ($\mathfrak{q}_i^m $ is a homogenous ideal, so we still get a direct sum decomposition of $k[x_{1}, \ldots ,x_i] / \mathfrak{q}_i^m $).
	Finally, because all monomials are nilpotent in this ring, this implies that $a $ or $b $ is in the ideal of nilpotents, showing that all non-units are nilpotent.
	Hence $\mathfrak{q}_i^m $ is primary.
\end{proof}

\begin{exercise}
	In a ring $A$, let $D(A)$ denote the set of prime ideals $\mathfrak{p}$ which satisfy the following condition: there exists $a \in A$ such that $\mathfrak{p}$ is minimal in the set of prime ideals containing $(0:a)$. Show that $x \in A$ is a zero divisor $\iff$ $x \in \mathfrak{p}$ for some $\mathfrak{p} \in D(A)$. \\
	Let $S$ be a multiplicatively closed subset of $A$, and identify $\Spec(S^{-1}A)$ with its image in $\Spec(A)$ (Chapter 3, Exercise 21). Show that
	\[
	D(S^{-1}A) = D(A) \cap \Spec(S^{-1}A).
	\]
	If the zero ideal has a primary decomposition, show that $D(A)$ is the set of associated prime ideals of $0$.
\end{exercise}
\begin{proof}
	Suppose that $x\in \mathfrak{p} $ for some $\mathfrak{p}\in D(A) $.
	Then because $a \mathfrak{p} \subseteq (0) $, $ax = 0 $ and $x $ is a zero divisor.
	
	If $x $ is a zero divisor: 
	Suppose $ax = 0 $.
	Then $x \in (0:a) $, and $(0:a) \ne (1) $ because $A $ can be assumed to be non-zero.
	Thus there is $\mathfrak{p} \supseteq (0:a) $.
	Because the intersection of a descending chain of prime ideals is prime, we can find a minimal such prime among those containing $(0:a) $.

	$D(S^{-1}A) = D(A) \cap \Spec(S^{-1}A)$:

	$\subseteq  $: The left hand side is the set of minimal prime ideals that contain some $(0:\frac{a}{b}) $.
	By the inclusion respecting bijection from Proposition 3.11, these correspond to minimal prime ideals in $A $ that don't meet $S $.
	Because each $S^{-1}\mathfrak{p} $ contains $(0:\frac{a}{b}) $, the contraction contains $(0:a) $ for if $ac = 0 $ for $c\in A $, then $\frac{a}{b}\cdot \frac{c}{1} = 0$ in $S^{-1}A $.

	$\supseteq  $: The right hand side is the set of minimal prime ideal that doesn't contain $S $ by Proposition 3.11 yet contain $(0:a) $ for some $a $ (it is minimal among these primes because the bijection $\Spec S^{-1}A $ to a subset of $\Spec A $ respects inclusions).
	Thus these biject to minimal prime ideals of $S^{-1} $, and they contain $(0:\frac{a}{1}) $ because $\forall \frac{p}{q} \in S^{-1}\mathfrak{p}, \frac{a}{1}\frac{p}{q} = \frac{ap}{q} = 0$ because $p\in \mathfrak{p} $ and $ap = 0 $.

	Finally, assume that there is a primary decomposition, say $0 = \cap \mathfrak{q}_i $ with associated primes $\mathfrak{p}_i = r(0:a_i)$.
	Take $\mathfrak{p} \in D(A) $ and let $\mathfrak{p}$ be minimal among prime ideals containing $(0:a)$.
	Then $\mathfrak{p} \supseteq r(0:a) $ because $r(0:a) $ is the intersection of prime ideals containing $(0:a) $.
	Thus
	\[
		\mathfrak{p} \supseteq r(\cap \mathfrak{q}_i:a) = \cap r(\mathfrak{q}_i:a) \supseteq r(0:a) \supseteq (0:a)
	.\]
	The latter containment is due to $0 $ being in all the $\mathfrak{q}_i $ so that $(\mathfrak{q}_i:a) \supseteq (0:a) $.

	We can see that $\cap r(\mathfrak{q}_i:a) = \cap_{\mathfrak{n}_i \in S \subseteq \{\mathfrak{p}_i\}  } \mathfrak{n}_i$ where $S $ is some subset of the associated primes.
	This is because for each $i $, $a \in \mathfrak{q}_i $ implies that $(\mathfrak{q}_i:a) =A$, allowing us to remove it from the intersection, and if $a\notin \mathfrak{q}_i $, then because $\mathfrak{q}_i $ is primary, $x\in (\mathfrak{q}_i:a) \implies ax \in \mathfrak{q}_i \implies x^a \in \mathfrak{q}_i $ for some $a $.
	Thus by taking the radical of $(\mathfrak{q}_i:a) $, we see that $x\in \mathfrak{p}_i $.

	Next, by Proposition 1.11, $\mathfrak{p} \supseteq \mathfrak{p}_i $ for some $i $.
	The minimality of $\mathfrak{p} $ then implies that $\mathfrak{p} = \mathfrak{p}_i $, an associated prime.

	Now conversely, an associated prime $\mathfrak{p}_i = r(0:a_i) $ is minimal among primes containing $(0:a_i) $ by definition of the radical.
\end{proof}

\begin{exercise}
	For any prime ideal $\mathfrak{p}$ in a ring $A$, let $S_\mathfrak{p}(0)$ denote the kernel of the homomorphism $A \to A_\mathfrak{p}$. Prove that 
	\begin{enumerate}
		\item $S_\mathfrak{p}(0) \subseteq \mathfrak{p}$.
		\item $r\big(S_\mathfrak{p}(0)\big) = \mathfrak{p} \iff \mathfrak{p}$ is a minimal prime ideal of $A$.
		\item If $\mathfrak{p} \supseteq \mathfrak{p}'$, then $S_\mathfrak{p}(0) \subseteq S_{\mathfrak{p} '}(0)$.
		\item $\cap_{\mathfrak{p} \in D(A)} S_\mathfrak{p}(0) = 0$, where $D(A)$ is defined in Exercise 9.
	\end{enumerate}
\end{exercise}
\begin{proof}
	i) Any $a \in S_ \mathfrak{p}(0) $ has the property that $\frac{a}{1} = 0 $, i.e. $\exists s\in A \setminus \mathfrak{p} $ such that $as = 0$ in $A $.
	Mapping this into $A / \mathfrak{p} $, we have that $as \equiv 0 $.
	Because $A / \mathfrak{p} $ is an integral domain and $s \notin \mathfrak{p} $, $a\equiv 0 \implies a \in \mathfrak{p}$.

	iii) Let $a \in S_ \mathfrak{p}(0) $.
	Then there is $s \in A \setminus \mathfrak{p} \subseteq A \setminus \mathfrak{p}' $ such that $as = 0 $ in $A $.
	But then $\frac{a}{1} = 0 \in A_{\mathfrak{p}'} $ because $A \setminus \mathfrak{p} \subseteq A \setminus \mathfrak{p}' $, so we can use the same element to prove that it vanishes.

	ii)

	$\implies) $ Suppose we have $\mathfrak{q} \subseteq \mathfrak{p} $.
	Then by iii) $S_{\mathfrak{p}}(0) \subseteq S_{\mathfrak{q}}(0)$.
	So we then have
	\[
		S_{\mathfrak{p}}(0) \subseteq S_{\mathfrak{q}}(0) \subseteq \mathfrak{q} \subseteq \mathfrak{p}
	.\] 
	But because $r(S_{\mathfrak{p}}(0)) = \mathfrak{p} $ and the radical is the intersection of the prime ideals containing it, this implies that $\mathfrak{p}\subseteq \mathfrak{q} $.
	Thus $\mathfrak{p} = \mathfrak{q} $ and $\mathfrak{p} $ is minimal.

	$\Leftarrow $)
	Because $\mathfrak{p} $ is minimal, $S \setminus \mathfrak{p} $ is maximal among multiplicatively closed subsets of $A $ that don't contain $0 $.
	Thus for any $x \in \mathfrak{p} $, $0 \in $ the multiplicatively closed subset spanned by $S \cup \{x\}   $.
	Hence there is $s\in S $ and $n $ such that $sx^n = 0 $.
	Thus $x \in r(0:s) $.
	But $S_{\mathfrak{p}}(0) = \cup_{s \in A \setminus \mathfrak{p}} (0:s) $ by Proposition 3.11.
	So by the remark on page 9, $r(S_{\mathfrak{p}}(0)) = \cup r(0:s) \implies x \in r(S_{\mathfrak{p}}(0)) $.
	As $x $ was arbitrary, $\mathfrak{p}\subseteq r(S_{\mathfrak{p}}(0)) $, which by exercise i) gives us that $\mathfrak{p} \subseteq r(S_{\mathfrak{p}}(0)) \subseteq r(\mathfrak{p}) = \mathfrak{p} $.
	Thus $r(S_{\mathfrak{p}}(0)) = \mathfrak{p} $.

	iv) Obviously $0 $ is in the intersection.
	Now suppose we have $x \ne 0$.
	Then $(0:x) \ne (1) $, so we can take a minimal prime ideal $\mathfrak{q} $ containing $(0:x) $.
	This is in $D(A) $ by definition.
	Because $(0:x) \subseteq \mathfrak{q} $, there is no $s \in A \setminus \mathfrak{q} $ such that $xs = 0 $ by definition of $(0:x) $.
	Thus $x \notin S_{\mathfrak{q}}(0) $.
	Hence $x\notin \cap S_{\mathfrak{p}}(0)$.
\end{proof}

\begin{exercise}
	If $\mathfrak{p}$ is a minimal prime ideal of a ring $A$, show that $S_\mathfrak{p}(0)$ (Exercise 10) is the smallest $\mathfrak{p}$-primary ideal. \\
	Let $\mathfrak{a}$ be the intersection of the ideals $S_\mathfrak{p}(0)$ as $\mathfrak{p}$ runs through the minimal prime ideals of $A$. Show that $\mathfrak{a}$ is contained in the nilradical of $A$.\\
	Suppose that the zero ideal is decomposable. Prove that $\mathfrak{a} = 0$ if and only if every prime ideal of $0$ is isolated.
\end{exercise}
\begin{proof}
	If $\mathfrak{p} $ is a minimal prime, then $r(S_ \mathfrak{p}(0)) = \mathfrak{p} $ by Exercise 4.10ii.
	Further, $S_{\mathfrak{p}}(0) $ is a primary ideal for if we have $ab \in S_{\mathfrak{p}}(0) $, then there is $c \in A \setminus \mathfrak{p} $ such that $abc = 0 $.
	As $0\in \mathfrak{p} $ and $\mathfrak{p} $ is prime, $c\notin \mathfrak{p} \implies ab \in \mathfrak{p} $.
	Thus either $a $ or $b $ are in $\mathfrak{p} = r(S_{\mathfrak{p}}(0)) $.
	Finally, $S_{\mathfrak{p}}(0) $ is the smallest $\mathfrak{p} $-primary ideal because given a $\mathfrak{p} $-primary ideal $I $, $I \subseteq r(I) = \mathfrak{p} $.
	Thus $A \setminus \mathfrak{p} \cap I = \emptyset $, so $I_{\mathfrak{p}} $ is $\mathfrak{p}_{\mathfrak{p}}$-primary by Proposition 4.8, and the contraction of $I_{\mathfrak{p}} $ is $I $.
	But the contraction contains $S_{\mathfrak{p}}(0) $.

	Because $S_{\mathfrak{p}}(0) \subseteq r(S_{\mathfrak{p}}(0)) $, we have that $\mathfrak{a} \subseteq r(\mathfrak{a}) = \cap r(S_{\mathfrak{p}_i}(0)) $ where the $\mathfrak{p}_i$ range over minimal prime ideals.
	But by Exercise 4.10ii, $r(S_{\mathfrak{p}_i}(0)) = \mathfrak{p}_i $.
	Thus $\mathfrak{a} \subseteq \cap \mathfrak{p}_i $.
	The RHS equals the nilradical because the nilradical is the intersection of all prime ideals, and we can just remove non-minimal ones.

	Assume that the zero ideal is decomposable into $\cap \mathfrak{q}_i$ with $r(\mathfrak{q}_i) = \mathfrak{p}_i $.

	Assume every prime ideal of $0 $ is isolated.
	Then no 
	By Exercise 4.9, $D(A) $ is the set of associated primes of $0 $, which by assumption are isolated.
	So by Exercise 4.10iv, $\cap_{\mathfrak{q}\in D(A)} S_{\mathfrak{q}}(0) = 0 $.

	Suppose that $\mathfrak{a} = 0 $.
	For each minimal prime $\mathfrak{p}^j $, let $\{\mathfrak{p}_{j,i}\}   $ meet $A \setminus \mathfrak{p}^j $.
	Let $\mathfrak{q}_{j,i} $ be the corresponding primary components.
	By Proposition 4.9, $\mathfrak{a} = \cap_{j} \cap_{j,i} \mathfrak{q}_{j,i} $.
	As this contains 
	Then because of Exercise 4.10iii), $\cap S_{\mathfrak{q}}(0) \subseteq \cap S_{\mathfrak{p}}(0) $ where $\mathfrak{q} $ range over associated primes of $0 $ as each $\mathfrak{q} $ contains a minimal prime ideal.
	Thus $\cap S_{\mathfrak{q}}(0) = 0 $.
\end{proof}

\begin{exercise}
	Let $A$ be a ring, $S$ a multiplicatively closed subset of $A$. For any ideal $\mathfrak{a}$, let $S(\mathfrak{a})$ denote the contraction of $S^{-1}\mathfrak{a}$ in $A$. The ideal $S(\mathfrak{a})$ is called the \textit{saturation} of $\mathfrak{a}$ with respect to $S$. Prove that 
	\begin{enumerate}
		\item $S(\mathfrak{a}) \cap S(\mathfrak{b}) = S(\mathfrak{a} \cap \mathfrak{b})$. 
		\item $S\big(r(\mathfrak{a})\big) = r\big(S(\mathfrak{a})\big)$. 
		\item $S(\mathfrak{a}) = (1) \iff \mathfrak{a}$ meets $S$.
		\item $S_1\big(S_2(\mathfrak{a})\big) = (S_1 S_2)(\mathfrak{a})$. \\
			If $\mathfrak{a}$ has a primary decomposition, prove that the set of ideals $S(\mathfrak{a})$ (where $S$ runs through all multiplicatively closed subsets of $A$) is finite.
	\end{enumerate}
\end{exercise}
\begin{proof}
	i) Because of Exercise 1.18, $S(\mathfrak{a})\cap S(\mathfrak{b}) = \mathfrak{a}^c \cap \mathfrak{b}^c = (\mathfrak{a} \cap \mathfrak{b})^c = S(\mathfrak{a}\cap \mathfrak{b}) $.

	ii) Because of Exercise 1.18, $S(r(\mathfrak{a})) = r(\mathfrak{a})^c = r(\mathfrak{a}^c) = r(S(\mathfrak{a}))$.
	
	iii) Proposition 3.11.

	iv) By Proposition 3.11ii), $S_{2}(\mathfrak{a}) = \mathfrak{a}^{ec} = (S_{2}^{-1}\mathfrak{a})^c = \cup_{s_{2} \in S_{2}} (\mathfrak{a}:s_{2})$..
	Thus $S_{1}(S_{2}(\mathfrak{a})) = \cup_{s_{1}\in S_{1}} \cup_{s_{2}\in S_{2}}((\mathfrak{a}:s_{2}):s_{1})$.
	By Exercise 1.12, this equals $\cup_{s_{1}\in S_{1}} \cup_{s_{2}\in S_{2}}(\mathfrak{a}:s_{2}s_{1}) $.
	This then equals $\cup_{s\in S_{1} S_{2}}(\mathfrak{a}:s) = ((S_{1}S_{2})^{-1} \mathfrak{a})^c = S_{1}S_{2}(\mathfrak{a})$.

	Now suppose that $\mathfrak{a} $ has a decomposition $\cap \mathfrak{q}_i $.
	Then by Proposition 4.9, $S(\mathfrak{a}) $ is an intersection of a finite subset of the $\mathfrak{q}_i $'s.
	There are only finitely many possibilities.
\end{proof}

\begin{exercise}
	Let $A$ be a ring and $\mathfrak{p}$ a prime ideal of $A$. Then \textit{$n$th symbolic power of} $\mathfrak{p}$ is defined to be the ideal (in the notation of Exercise 12)
	\[
		\mathfrak{p}^{(n)} = S_\mathfrak{p}(\mathfrak{p}^n)
	\]
	where $S_\mathfrak{p} = A \setminus \mathfrak{p}$. Show that
	\begin{enumerate}
		\item $\mathfrak{p}^{(n)}$ is a $\mathfrak{p}$-primary ideal;
		\item if $\mathfrak{p}^n$ has a primary decomposition, then $\mathfrak{p}^{(n)}$ is its $\mathfrak{p}$-primary component;
		\item If $\mathfrak{p}^{(m)}\mathfrak{p}^{(n)}$ has a primary decomposition, then $\mathfrak{p}^{(m+n)}$ is its $\mathfrak{p}$-primary component
		\item $\mathfrak{p}^{(n)} = \mathfrak{p}^n \iff \mathfrak{p}^{n}$ is $\mathfrak{p}$-primary.
	\end{enumerate}
\end{exercise}
\begin{proof}
	i) Suppose we have $ab \in \mathfrak{p}^{(n)}  $ and $a\notin \mathfrak{p}^{(n)}  $.
	Then by Proposition 3.11, $S_{\mathfrak{p}}(\mathfrak{p}^n) = \cup_{s \in A \setminus \mathfrak{p}} (\mathfrak{p}^{n} :s) $.
	Now let $s $ be such that $ab \in (\mathfrak{p}^n:s) $.
	Then $ab s \in \mathfrak{p}^n$.

	If $bs \in \mathfrak{p} $, then because $s\notin \mathfrak{p} $, $b \in \mathfrak{p} \implies b^n \in \mathfrak{p}^{(n)}  $, the requirement for $\mathfrak{p}^{(n)}  $ to be primary.
	If $bs \notin \mathfrak{p} $, then $a\in (\mathfrak{p}^n:bs) \subseteq \cup_{s\in A \setminus \mathfrak{p}} (\mathfrak{p}^n: s) $.

	Finally, because $r((S_{\mathfrak{p}}\mathfrak{p}^n)^c) = r(S_{\mathfrak{p}}\mathfrak{p}^n)^c = (S_{\mathfrak{p}}\mathfrak{p})^c $ because of Proposition 4.8.
	Then by Proposition 3.11 this equals $\mathfrak{p} $.

	Alternatively, one can observe that $\mathfrak{p} $ is minimal in $A / \mathfrak{p}^n $	 because $r(\mathfrak{p}^n) = \mathfrak{p} $ and $r(\mathfrak{p}^n)$ is contained in every prime ideal containing $\mathfrak{p}^n $ and every ideal of $A/\mathfrak{p}^n$ corresponds to an ideal of $A $ containing $\mathfrak{p}^n $.
	Thus Exercise 1.11 applied to $A / \mathfrak{p}^n $ with the minimal prime $\mathfrak{p} $ gives us that $S_p(0) $ is the smallest $\mathfrak{p} $ primary ideal in $A / \mathfrak{p}^n $.
	This corresponds to the smallest $\mathfrak{p} $ primary ideal in $A $ containing $\mathfrak{p}^n $.

	ii) 
	Take a minimal decomposition of $\mathfrak{p}^n $.
	Because $r(\mathfrak{p}^n:1) = \mathfrak{p} $ and of the uniqueness theorem, there are $\mathfrak{p} $-primary ideals in the decomposition.
	Recall that the alternate proof above tells us that $\mathfrak{p}^{(n)}$ is the smallest $\mathfrak{p} $-primary ideal containing $\mathfrak{p}^n $.
	As any primary ideal in the decomposition must contain $\mathfrak{p}^n $, $\mathfrak{p}^{(n)}  $ must be contained in any $\mathfrak{p} $-primary associated ideal to $\mathfrak{p}^n $.
	But this contradicts minimality unless the $\mathfrak{p} $-primary associated ideal was $\mathfrak{p}^{(n)}$. 

	iii) 
	% Take a minimal decomposition of $\mathfrak{p}^{(m)}\mathfrak{p}^{(n)}   $.
	% Observe that $(\mathfrak{p}^{(m)}\mathfrak{p}^{(n)})^e = (\mathfrak{p}^{(m)})^e (\mathfrak{p}^{(n)})^e $ by Exercise 1.18.
	% Then because $\mathfrak{p}^{(n)} = \mathfrak{p}^{ec}   $ and of Proposition 1.17, $(\mathfrak{p}^{(m)}\mathfrak{p}^{(n)})^e = (\mathfrak{p}^{m})^e (\mathfrak{p}^n)^e = (\mathfrak{p}^{m+n})^e $.
	%
	% By applying Proposition 4.9 with $S = A \setminus \mathfrak{p} $, we get that
	% \[
	% 	(\mathfrak{p}^{(m)}\mathfrak{p}^{(n)})^e = \cap \mathfrak{q}_i^e
	% \]
	% where $\mathfrak{q}_i $ range over the primary ideals that don't meet $S $.
	% From above we then have
	% \[
	% 	(\mathfrak{p}^{m+n})^e = (\mathfrak{p}^{(m)}\mathfrak{p}^{(n)})^e = \cap \mathfrak{q}_i^e
	% \]
	% By ii, $\mathfrak{p}^{(m+n)}  $ is the $\mathfrak{p} $-primary component of $\mathfrak{p}^{m+n}  $.
	% Thus when we contract, we get $\mathfrak{p}^{(m+n)}  $ as the $\mathfrak{p} $-primary component on the left and the intersection of primary ideals whose radical contains $\mathfrak{p} $.
	% By the uniqueness theorem of 
	Because $\mathfrak{p}^{m} \subseteq \mathfrak{p}^{(m)}   $, $\mathfrak{p}^{m+n} \subseteq \mathfrak{p}^{(m)}\mathfrak{p}^{(n)}    $.
	Thus all the $\mathfrak{p} $-primary ideals in the decomposition contain $\mathfrak{p}^{m+n}  $, and thus contain $\mathfrak{p}^{(m+n)}  $ as it is the smallest such.

	Finally, we can show that $\mathfrak{p}^{(m)}\mathfrak{p}^{(n)}\subseteq \mathfrak{p}^{(m+n)}$, which completes the problem because we can then intersect a minimal decomposition of $\mathfrak{p}^{(m)}\mathfrak{p}^{(n)}   $ with $\mathfrak{p}^{(m+n)}  $ and collect the $\mathfrak{p}$-primary ideals together via Lemma 4.3 to get that $\mathfrak{p}^{(m+n)}  $ is the $\mathfrak{p} $-primary component.
	Now to show this: by definition, $\mathfrak{p}^{(m)}\mathfrak{p}^{(n)} = (\mathfrak{p}^m)^{ec}(\mathfrak{p}^n)^{ec}$.
	Thus by taking extensions and using Exercise 1.18 and Proposition 1.17, we have that $(\mathfrak{p}^{m+n})^e = (\mathfrak{p}^{(m)}\mathfrak{p}^{(n)})^e$.
	Contracting, we get that
	\[
		\mathfrak{p}^{(m+n)} = ((\mathfrak{p}^{(m)})^e (\mathfrak{p}^{(n)})^e))^c \supseteq (\mathfrak{p}^{(m)})^{ec} (\mathfrak{p}^{(n)})^{ec} \supseteq \mathfrak{p}^{(m)}\mathfrak{p}^{(n)}  
	\] 
	via assorted uses of Proposition 1.17 and Exercise 1.18.

	iv) If $\mathfrak{p}^{(n)} = \mathfrak{p}^n  $, then we are done by part i.
	If $\mathfrak{p}^n $ is $\mathfrak{p} $-primary, then because $\mathfrak{p}^{(n)}  $ is the smallest $\mathfrak{p} $ primary ideal containing $\mathfrak{p}^n $ by the alternate proof of i) and $\mathfrak{p}^n $ is a $\mathfrak{p} $-primary ideal, $\mathfrak{p}^{(n)} \subseteq \mathfrak{p}  $.
	But clearly $\mathfrak{p}^{(n)} \supseteq \mathfrak{p}  $.
\end{proof}

\begin{exercise}
	Let $\mathfrak{a}$ be a decomposable ideal in a ring $A$ and let $\mathfrak{p}$ be a maximal element of the set of ideals $(\mathfrak{a} : x)$, where $x \in A$ and $x \notin \mathfrak{a}$. Show that $\mathfrak{p}$ is a prime ideal belonging to $\mathfrak{a}$. 
\end{exercise}
\begin{proof}
	First we can see that $\mathfrak{p} $ is prime: suppose we have $ab \in \mathfrak{p} = (\mathfrak{a}:x) $.
	Because $\mathfrak{a} $ is decomposable, $(\mathfrak{a}:x) = (\cap \mathfrak{q}_i:x) = \cap (\mathfrak{q}_i:x) $.
	If $a\notin \mathfrak{p} $, then $(\mathfrak{p}:a) \supseteq \mathfrak{p} $.
	But then $(\mathfrak{p}:a) = (\cap (\mathfrak{q}_i:x): a) = \cap ((\mathfrak{q}_i:x):a) = \cap (\mathfrak{q}_i:ax) = (\cap \mathfrak{q}_i:ax) = (\mathfrak{a}:ax)$ by Exercise 1.12 in the chapter.

	If $ax\notin \mathfrak{a} $, then because $\mathfrak{p} $ is maximal $(\mathfrak{p}:a) = \mathfrak{p} $, and $b \in (\mathfrak{p}:a) $.
	If $ax \in \mathfrak{a} $, then by definition, $a \in (\mathfrak{a}:x) $.

	Finally, $\mathfrak{p} $ is a prime ideal belonging to $\mathfrak{a} $ because it is of the right form for the Uniqueness Theorem: $\mathfrak{p} = (\mathfrak{a}:x) = r(\mathfrak{a}:x) $.
\end{proof}

\begin{exercise}
	Let $\mathfrak{a}$ be a decomposable ideal in a ring $A$, let $\Sigma$ be an isolated set of prime ideals belonging to $\mathfrak{a}$, and let $\mathfrak{q}_\Sigma$ be the intersection of the corresponding primary components. Let $f\!$ be an element of $A$ such that, for each prime ideal $\mathfrak{p}$ belonging to $\mathfrak{a}$, we have $f \!\in \mathfrak{p} \iff \mathfrak{p} \notin \Sigma$, and let $S_f\!$ be the set of all powers of $f\!$. Show that $\mathfrak{q}_\Sigma = S_f (\mathfrak{a}) = (\mathfrak{a} : f^n)$ for all large $n$. 
\end{exercise}
\begin{proof}
	First we can show that $S_f(\mathfrak{a}) = (\mathfrak{a}:f^{n})  $
	We use Proposition 3.11 to get that $\mathfrak{a}^{ec} = \cup_{n}(\mathfrak{a}:f^n)  $.
	Clearly $(\mathfrak{a}:f^n) \subseteq (\mathfrak{a}:f^{n+1})  $.
	Thus if we show that $(\mathfrak{a}:f^n) $ stabilizes for some large enough $n $, we have shown this part.
	Now 

	Take some $q \in \mathfrak{q}_\Sigma $.
	Then $q \in \mathfrak{q}_i $ for some isolated component
\end{proof}

\begin{exercise}
	If $A$ is a ring in which every ideal has a primary decomposition, show that every ring of fractions $S^{-1}A$ has the same property. 
\end{exercise}
\begin{proof}
	Just Proposition 4.9 and Proposition 3.11.
\end{proof}

\begin{exercise}
Let $A$ be a ring with the following property.

(L1) For every ideal $\mathfrak{a} \neq (1)$ in $A$ and every prime ideal $\mathfrak{p}$, there exists $x\notin \mathfrak{p}$ such that $S_\mathfrak{p}(\mathfrak{a}) = (\mathfrak{a} : x)$, where $S_\mathfrak{p} = A \setminus \mathfrak{p}$.

Then every ideal in $A$ is an intersection of (possibly infinitely many) primary ideals.
\ifhint
	[Let $\mathfrak{a} $ be an ideal $\ne (1) $ in $A $, and let $\mathfrak{p}_1 $ be a minimal element of the set of prime ideals containing $\mathfrak{a} $. Then $\mathfrak{q}_1 = S_{\mathfrak{p}_1}(\mathfrak{a}) $ is $\mathfrak{p}_1 $-primary (by Exercise 11), and $\mathfrak{q}_1 = (\mathfrak{a}:x) $ for some $x\notin \mathfrak{p}_1 $. Show that $\mathfrak{a} = \mathfrak{q}_1 \cap (\mathfrak{a}+((x)) $.

	Now let $\mathfrak{a}_1$ be a maximal element of the set of ideals $\mathfrak{b} \supseteq \mathfrak{a}$ such that $\mathfrak{q}_1 \cap \mathfrak{b} = \mathfrak{a}$, and choose $\mathfrak{a}_1$ so that $x \in \mathfrak{a}_1$, and therefore $\mathfrak{a}_1 \not\subseteq \mathfrak{p}_1$. Repeat the construction starting with $\mathfrak{a}_1$ and so on. At the $n$th stage we have $\mathfrak{a} = \mathfrak{q}_1 \cap \cdots \cap \mathfrak{q}_n \cap \mathfrak{a}_n$ where the $\mathfrak{q}_1$ are primary ideals, $\mathfrak{a}_n$ is maximal among the ideals $\mathfrak{b}$ containing $\mathfrak{a}_{n -1} = \mathfrak{a}_n \cap \mathfrak{q}_n$ such that $\mathfrak{a} = \mathfrak{q}_1 \cap \cdots \cap \mathfrak{q}_n \cap \mathfrak{b}$, and $\mathfrak{a}_n \not\subseteq \mathfrak{p}_n$. If at any stage we have $\mathfrak{a}_n = (1)$, the process stops, and $\mathfrak{a}$ is a finite intersection of primary ideals. If not, continue by transfinite induction, observing that each $\mathfrak{a}_n$ strictly contains $\mathfrak{a}_{n-1} $.]
\fi
\end{exercise}
\begin{proof}
	
\end{proof}
