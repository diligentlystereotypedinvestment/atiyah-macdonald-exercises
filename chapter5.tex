% root = exercises.tex
\section{Integral Dependence and Valuations}

\begin{exercise}
	Let $f \colon A \to B$ be an integral homomorphism of rings. Show that $f^\ast\colon \Spec(B) \to \Spec(A)$ is a \textit{closed} mapping, i.e. that it maps closed sets to closed sets. (This is a geometrical equivalent of (5.10).)
\end{exercise}
\begin{proof}
	We want to show that for all ideals $\mathfrak{a} \subseteq B $, $f^\ast(V(\mathfrak{a})) = V(\mathfrak{b}) $ for some ideal $\mathfrak{b} \subseteq A $.
	I propose that we let $\mathfrak{b} = f^{-1}(\mathfrak{a}) $ (which is an ideal because preimages take ideals to ideals).
	
	$\subseteq  $: For $\mathfrak{p} \in V(\mathfrak{a}) $, $\mathfrak{a} \subseteq \mathfrak{p} $.
	Thus $\mathfrak{b} = f^{-1}(\mathfrak{a}) \subseteq f^{-1}(\mathfrak{p}) = f^\ast(\mathfrak{p}) \implies f^\ast(\mathfrak{p}) \in V(\mathfrak{b})$.

	$\supseteq  $: Take some $\mathfrak{p} \in V(\mathfrak{b}) $.
	We can first note that because $\mathfrak{b} = f^{-1}(\mathfrak{a}) $ and $0\in \mathfrak{a} $, $\ker f \subseteq \mathfrak{b} \subseteq \mathfrak{p} $.
	Then $f(\mathfrak{p}) $ is prime as an ideal of $f(A) $, because if $f(a)f(b) \in f(\mathfrak{p}) $, then $f(ab) \in f(\mathfrak{p})$ implies that there is some $c \in \ker f $ such that $ab - c \in \mathfrak{p} $.
	Because $\ker f \subseteq \mathfrak{p} $, $ab \in \mathfrak{p}\implies $ either $a $ or $b $ is in $\mathfrak{p} $, showing that $f(\mathfrak{p}) $ is prie.

	Then by Theorem 5.10, as $B $ is integral over $f(A) $, $\exists \mathfrak{q} \in \Spec B $ such that $\mathfrak{q}\cap f(A) = f(\mathfrak{p})$.
	Finally, $f^{-1}(\mathfrak{q}) = \mathfrak{p}$ shows that this map is surjective.
\end{proof}

\begin{exercise}
	Let $A$ be a subring of a ring $B$ such that $B$ is integral over $A$, and let $f \colon A \to \Omega$ be a homomorphism of $A$ into an algebraically closed field $\Omega$. Show that $f\!$ can be extended to a homomorphism of $B$ into $\Omega$.
\end{exercise}
\begin{proof}
	Let $\Sigma $ be the set of all rings $R $ such that $A \subseteq R \subseteq B $ and $R $ has an extension of $f $ to $R $.
	Give it a partial order of $R_{1} \le R_{2} $ iff $R_{1}\subseteq R_{2} $ and $f_{R_{2}}$ is an extension of $f_{R_{1}} $.
	Every ascending chain has bounded above since given a sequence $R_{1} \subseteq \cdots \subseteq R_{2} \subseteq \cdots  $ in $\Sigma $, their union is a subring of $B $ and $f $ can be extended to the union by mapping $x \in R_i $ to $f_{R_i}(x) $.
	This is well-defined because the partial order ensures the output doesn't depend on the choice of $i $.
	Thus by Zorn's Lemma, we have a maximal element $R $.

	This maximal element is $B $, because given an element $x \in B $, $R[x] $ a subring of $B $ and has an extension of $f $ to $R[x] $ as follows:
	Suppose $x $ satisfies the equation
	\[
		x^n + a_1 x^{n-1} + \cdots + a_n = 0
	.\] 
	Then map $x $ to a root of this polynomial in $\Omega $ and extend $f_R $ in this way.
	Because $R$ is maximal, $R = R[x] \implies x \in R $.
	Thus $B \subseteq R \subseteq B \implies R = B $.
\end{proof}

\begin{exercise}
	Let $f \colon B \to B'$ be a homomorphism of $A$-algebras, and let $C$ be an $A$-algebra. If $f$ is integral, prove that $f \otimes 1\colon B \otimes_A C \to B' \otimes_A C$ is integral. (This includes (5.6) ii) as a special case.)
\end{exercise}
\begin{proof}
	It suffices to show that pure tensors are integral over $\im(f \otimes 1) $ because they generate $B' \otimes _A C $ and sums and products of integral elements are integral (i.e. Corollary 5.3).
	Now suppose we have $b' \otimes c $ and
	\[
		(b')^n + a_1 (b')^{n-1} + \cdots + a_n = 0 \qquad a_i \in f(A)
	\] 
	because $f $ is integral.

	Then we will show that
	\[
		(b' \otimes c)^n + a_1(1 \otimes c) (b' \otimes c)^{n-1} + \cdots + a_n(1 \otimes c^n)(1 \otimes 1)
	\] 
	will be an integral equation for $b' \otimes c $ over $f(B \otimes_A C) $.
	First, each $a_i(1 \otimes c^i) $ is in $f(B \otimes_A C) $ as $a_i \in f(B) $.
	Then we expand:
	\begin{align*}
		(b' \otimes c)^n &+ a_1(1 \otimes c) (b' \otimes c)^{n-1} + \cdots + a_n(1 \otimes c^n)(1 \otimes 1)\\
		&= ((b')^n \otimes c^n) + (a_{1} (b')^{n-1}  \otimes c^{n-1} ) + \cdots + (a_n \otimes c^n)\\
		&= ((b')^n + a_1 (b')^{n-1} + \cdots + a_n) \otimes c^n\\
		&= 0 \otimes c^n = 0
	\end{align*}
\end{proof}

\begin{exercise}
	Let $A$ be a subring of $B$ such that $B$ is integral over $A$. Let $\mathfrak{n}$ be a maximal ideal of $B$ and let $\mathfrak{m} = \mathfrak{n} \cap A$ be the corresponding maximal ideal of $A$ (see (5.8)). Is $B_{\mathfrak{n}}$ necessarily integral over $A_\mathfrak{m}$?
\end{exercise}
\begin{proof}
	This is not true.
	Fix a field $k $ and consider $B = k[x] $ and $A = k[x^2-1] $.
	Then $x $ is a root of $y^2 - (1 - (x^2-1)) \in A[y] $, so $B $ is integral over $A $ .

	Finally, let $\mathfrak{n} = (x-1) $.
	Then $\mathfrak{m} = (x^2-1) $.
	Now I claim that $\frac{1}{x+1} $ is not integral in $B_{\mathfrak{n}} $ over $A_{\mathfrak{m}} $.
	Suppose FTSOC that it was.
	Then we have $f(y) \in A_{\mathfrak{m}}[y] $ such that $f $ vanishes at $\frac{1}{x+1} $.
	% Write $f(y) = y^n + f_1y^{n-1} + \cdots + f_n  $.

	Let $F = A_{(0)} $ and $E = B_{(0)} $.
	Then by field theory, the minimal polynomial of $\frac{1}{x+1} $ divides $f(y) $ in $F[y]$.
	Its minimal polynomial is $y^2 + \frac{2}{x^2-1}y - \frac{1}{x^2-1} $.
	So suppose that
	\begin{equation}\label{eqn:5.4.1}
		f = (y^2+\frac{2}{x^2-1}y - \frac{1}{x^2-1})g
	\end{equation}
	for $g \in F[y] $.

	If $g $ has $x^2-1 $ terms in its (simplified) denominator, then we get an obvious contradiction with $f $ being in $A_{\mathfrak{m}}[y] $.
	If $g $ has no $x^2-1 $	terms in its (simplified) numerator, then we also get an obvious contradiction.
	Thus $g = (x^2-1)h$ for some $h \in F[y] $.
	But then the leading coefficient of $f $ (in $y $) will have an $x^2-1 $ term in it, contradicting the definition of $f $ being an integral relation for $\frac{1}{x+1} $.
\end{proof}

\begin{exercise}
	Let $A \subseteq B$ be rings, $B$ integral over $A$.
	\begin{enumerate}
		\item If $x \in A$ is a unit in $B$ then it is a unit of $A$.
		\item The Jacobson radical of $A$ is the contraction of the Jacobson radical of $B$.
\end{enumerate}
\end{exercise}
\begin{proof}
	i) Suppose we have $xb = 1 $ with $b \in B $.
	Then we have some integral relation of lowest degree
	\[
		b^n + a_1 b^{n-1} + \cdots + a_n = 0
	\] 
	with $a_i \in A $.
	By multiplying it by $x $, we get
	\[
		b^{n-1} + a_1 b^{n-2} + \cdots + a_{n-1} + a_nx = 0
	\] 
	But this is an integral relation of lower degree, implying that $b $ is the root of a polynomial over $A $ with degree 1.
	But this just implies that $b \in A $.

	ii) The contraction of the Jacobson radical of $B $ is $(\bigcap_{\mathfrak{m}\in\Specm B} \mathfrak{m}) \cap A $.
	By Corollary 5.8, $\mathfrak{m} \in \Specm B \iff \mathfrak{m}\cap A $ is Maximal.
	Thus $(\bigcap_{\mathfrak{m}\in \Specm B} \mathfrak{m})\cap A = \bigcap_{\mathfrak{n}\in\Specm A} \mathfrak{n}$, which is the Jacobson radical of $A $.
\end{proof}

\begin{exercise}
	Let $B_1,\,\ldots,\,B_n$ be integral $A$-algebras. Show that $\prod_{i=1}^n B_i$ is an integral $A$-algebra.
\end{exercise}
\begin{proof}
	Suppose we have $(b_{1},b_{2},b_n) \in \prod B_i $ with
	\[
		f_{i}(b_{i}) = b_{i}^{n_i} + a_{1i}^{n_i-i} + \cdots + a_{n_ii} = 0.
	\]
	Then let $f(x) = \prod f_i(x)$.
	This is a polynomial over $A $, and by considering it as a polynomial in $\prod B_i$, we can see that $(b_{1},b_{2}, \ldots, b_n) $ is a root of it: the ring operation is done coordinate wise, so $f((b_{1},b_{2},\ldots,b_n)) = (f(b_{1}),f(b_{2}),\ldots,f(b_n)) = (0,0,\ldots,0) $.
\end{proof}

\begin{exercise}
	Let $A$ be a subring of a ring $B$, such that the set $B \setminus A$ is closed under multiplication. Show that $A$ is integrally closed in $B$.
\end{exercise}
\begin{proof}
	Suppose we have an integral relation for $b \in B \setminus A$ of least degree,
	\[
		b^n + a_1 b^{n-1} + \cdots + a_n = 0
	.\] 
	Then I claim that we get a contradiction from this:
	\[
		b(b^{n-1} + a_1 b^{n-2} + \cdots + a_{n-1}) + a_n = 0
	.\] 
	Suppose that $b^{n-1} + a_1 b^{n-2} + \cdots + a_{n-1} \in A$.
	Then by subtracting the element of $A $, we get a lower degree integral relation, a contradiction.
	Thus $b^{n-1} + a_1 b^{n-2} + \cdots + a_{n-1} \in B \setminus A$.
	Because $B\setminus A $ is multiplicatively closed, $b(b^{n-1} + a_1 b^{n-2} + \cdots + a_{n-1}) \in B \setminus A $.
	But $-a_n $ is in $A $, a contradiction.
\end{proof}

\begin{exercise}
	~
	\begin{enumerate}
		\item Let $A$ be a subring of an integral domain $B$, and let $C$ be the integral closure of $A$ in $B$. Let $f,\,g$ be monic polynomials in $B[x]$ such that $fg \in C[x]$. Then $f,\, g$ are in $C[x]$.
		\item Prove the same result without assuming that $B$ (or $A$) is an integral domain.
	\end{enumerate}
\end{exercise}
\begin{proof}
	ii)
	We use induction on the degree of $fg $.

	First we show it for $fg $ of degree $2$.
	Then $f(x) = x+a $ and $g(x) = x+b $ for some $a,b \in B $.
	If $fg \in C $, then $f(x)g(x) = x^2+(a+b)x+ab $.

	Using the quadratic formula shows us that $-a,-b $ are roots of this polynomial, so $-a,-b $ are integral over $A[a+b,ab]$.
	Because $a+b,ab \in C $, $A[a+b,ab] $ is finitely generated as an $A $-module.
	As $-a,-b $ are integral over $A[a+b,ab] $, $A[a+b,ab,-a,-b] $ is finitely generated as an $A$-module.
	Hence $-a,-b \in C $.
	Thus $f,g \in C[x] $.

	Finally, assume it is true up to $fg $ of degree $n-1 $.
	Let $f(x) = f_{1}(x)x + b_{1} $ and $g(x) = g_{1}(x)x + b_{2} $.
	Note that if $fg \in C[x] $, $(b_{1}g_{1}(x)+b_{2}f_{1}(x))x + b_{1}b_{2} \in C[x] $, being the last two terms of $fg $.
	Thus by subtracting off $(b_{1}g_{1}(x)+b_{2}f_{1}(x))x + b_{1}b_{2} $, the result is also in $C[x] $.
	Thus $f_{1}(x)g_{1}(x)x^2 \in C[x] $, so $f_{1}(x)g_{1}(x) \in C[x] $.
	By our induction hypothesis, $f_{1},g_{1} \in C[x] $.
	Thus $f_{1}(x)x+b_{1}, g_{1}(x)x+b_{2} $ are in $C[x] $.
\end{proof}

\begin{exercise}
	Let $A$ be a subring of a ring $B$ and let $C$ be the integral closure of $A$ in $B$. Prove that $C[x]$ is the integral closure of $A[x]$ in $B[x]$.
\end{exercise}
\begin{proof}
	We use induction on the degree to first show that $C[x] $ is integral over $A[x] $.
	Obviously degree 0 terms of $C[x] $ are integral over $A[x] $.
	We can also show that $x $ is integral over $A[x] $, since it is the root of $y - x $.

	Now assume that all terms of $C[x] $ up to degree $n-1 $ are integral over $A[x] $.
	Say we have $f = c_{0}x^n + \cdots + c_n $.
	By induction hypothesis, $c_{1}x^{n-1} + \cdots + c_n $ is integral over $A[x] $.
	Thus if we show that $f - (c_{1}x^{n-1} + \cdots + c_n) = c_{0}x^n $ is integral over $A[x]$, we are done.
	Finally, because $x $ and $c_{0} $ are integral over $A[x] $ and products of integral elements are integral, $c_{0}x^n $ is integral.

	Now FTSOC suppose we have $f = b_{0}x^n + \cdots + b_n \in B[x] \setminus C[x]$ integral over $A[x] $.
	We can pick one with the least number of non-zero coefficients.
	Then any integral relation of $f $ will produce an integral relation for $b_n $ by focusing on the degree 0 component.
	Thus $b_n \in C $.
	Hence $f-b_n $ is also integral over $A[x] $ and in $B[x] \setminus C[x] $.
	But $f-b_n $ has fewer non-zero coefficients than $f $, contradicting our assumption, allowing us to conclude that $C[x] $ is the integral closure of $A[x] $.
\end{proof}

\begin{exercise}
	A ring homomorphism $f \colon A \to B$ is said to have the \textit{going-up property} (resp. the \textit{going-down property}) if the conclusion of the going-up theorem (5.11) (resp. the going-down theorem (5.16)) holds for $B$ and its subring $f(A)$.

	Let $f^\ast \colon \Spec(B) \to \Spec(A)$ be the mapping associated with $f$.
	\begin{enumerate}
		\item Consider the following three statements:
		\begin{enumerate}
			\item $f^\ast$ is a closed mapping.
			\item $f$ has the going-up property.
			\item Let $\mathfrak{q}$ be any prime ideal of $B$ and let $\mathfrak{p} = \mathfrak{q}^c$. Then $f^\ast\colon \Spec(B/\mathfrak{q}) \to \Spec(A/\mathfrak{p})$ is surjective.
		\end{enumerate}
		Prove that (a) $\implies$ (b) $\iff$ (c) (See also Chapter 6, Exercise 11.)

		\item Consider the following three statements:
		\begin{enumerate}
			\item[(a')] $f^\ast$ is an open mapping.
			\item[(b')] $f$ has the going-down property.
			\item[(c')] For any prime ideal $\mathfrak{q}$ of $B$, if $\mathfrak{p} = \mathfrak{q}^c$, then $f^\ast\colon \Spec(B_\mathfrak{q}) \to \Spec(A_\mathfrak{p})$ is surjective.
		\end{enumerate}

		Prove that $(a')\implies(b')\iff(c')$. (See also Chapter 7, Exercise 23).
\end{enumerate}
\end{exercise}
\begin{proof}
	$(a) \implies (b) $ 
	By induction, it suffices to show the going up property for $n=2,m=1 $.
	So suppose we have $\mathfrak{p}_1 \subseteq \mathfrak{p}_2$ a chain of prime ideals of $f(A) $ and $\mathfrak{q}_1$ a prime ideal in $B $ with $\mathfrak{q}_1\cap f(A) = \mathfrak{p}_1 $.

	First note that $A / \ker f \cong \im f = f(A) $ by the first isomorphism theorem and $f: A\to B $ factors through $A / \ker f $.
	This gives us that $f^\ast: \Spec B \to \Spec A $ factors through $\Spec A / \ker f $ because $\Spec$ is a functor.
	Then because $\Spec A / \ker f $ is a closed subset of $\Spec A $ (Proposition 1.1), $f^\ast $ being a closed map gives us that the factored map $\Spec B \to \Spec A / \ker f $ is closed.

	Finally, this then tells us that $f^\ast(V(\mathfrak{q}_1)) = V(\mathfrak{a}) $.
	Because $f^\ast(\mathfrak{q}_1) = \mathfrak{p}_1 $, $\mathfrak{p}_1 \in V(\mathfrak{a}) $.
	Because $\mathfrak{p}_2 \supseteq \mathfrak{p}_1 $, $\mathfrak{p}_2 \in V(\mathfrak{a}) $.
	Thus there is some $\mathfrak{q}_2 \in V(\mathfrak{q}_1) $ such that $f^\ast(\mathfrak{q}_2) = \mathfrak{p}_2 $.
	This is the condition we need to go up.

	$(b) \iff (c) $ First we show the fowards direction.
	Take some prime $\mathfrak{n}' \in \Spec(A/(\mathfrak{p}) $.
	This corresponds to a prime ideal $\mathfrak{n} \in \Spec A $ that contains $\mathfrak{p} $.
	Thus it suffices to show that there is a prime ideal in $B $ that contains $\mathfrak{q} $ that contracts to $\mathfrak{n} $.

	Next we can see that $\mathfrak{p} $ and $\mathfrak{n} $ correspond to prime ideals $p,n $ in $f(A) $ as $\mathfrak{q}^c = \mathfrak{p} \implies \ker f \subseteq \mathfrak{p} $.
	So we have this set up:
	\[
		p \subseteq n
	\]
	with $\mathfrak{q}\cap f(A) = p$, so by going up, we have a prime ideal $\mathfrak{t} \in \Spec B $ such that $\mathscr{t} \cap f(A) = n $.
	Thus $\mathfrak{t} $ is a prime ideal in $B $ that contains $\mathfrak{q} $ and contracts to $\mathfrak{n} $, showing surjectivity.

	Now to show the reverse direction.
	It suffices to show the going up with $n=2,m=1$, so take $\mathfrak{p}_1 \subseteq \mathfrak{p}_2 $ prime ideals in $f(A) $ and $\mathfrak{q}_1 $ a prime ideal in $B $ with $\mathfrak{q}_1 \cap f(A) = \mathfrak{p}_1 $.
	Then let $\mathfrak{q} = \mathfrak{q}_1 $ and $\mathfrak{p}= f^{-1}(\mathfrak{p}_1) $.
	By hypothesis, $\Spec B / \mathfrak{q} \to \Spec A / \mathfrak{p} $ is surjective.

	Because $f^{-1}(\mathfrak{p}_2) \supseteq f^{-1}(\mathfrak{p}_1)$, by Proposition 1.1, $f^{-1}(\mathfrak{p}_2)$ corresponds to a prime ideal in $A / \mathfrak{p} $, denote it be $p_{2} $.
	Now let $q_{2} $ be the preimage of it in $\Spec B / \mathfrak{q} $.
	This corresponds to a prime ideal $\mathfrak{q}_2 $ in $B $ containing $\mathfrak{q} $ by Proposition 1.1.
	This is the desired extension as $\mathfrak{q}_2 \supseteq \mathfrak{q} = \mathfrak{q}_1 $ and $\mathfrak{q}_2 \cap f(A) = f(f^{-1}(\mathfrak{q}_2)) = f(f^\ast(\mathfrak{q}_2)) = f(f^{-1}(\mathfrak{p}_2)) = \mathfrak{p}_2$.

	$(a') \implies (b') $ It suffices to show the going down property for $\mathfrak{p}_1 \supseteq \mathfrak{p}_2 $ and $\mathfrak{q}_1 $ prime ideals of $f(A) $ and $B $ respectively such that $\mathfrak{q}_1 \cap f(A) = \mathfrak{p}_1 $.
\end{proof}
